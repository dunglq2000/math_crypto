\documentclass{beamer}
\usepackage[utf8]{inputenc}
\usepackage[T2A]{fontenc}
\usepackage[russian]{babel}
\usepackage{amsmath,amssymb,amsthm,empheq}
\usepackage{appendixnumberbeamer}
\usepackage{tikz,circuitikz}
\usepackage{color, colortbl}
\usepackage{multicol}

\usetheme{Boadilla}
\usecolortheme{orchid}
\setbeamertemplate{footline}[frame number]
\beamertemplatenavigationsymbolsempty


\begin{document}
	\tikzstyle{block} = [rectangle,	draw, minimum height=1cm,	text centered]
	\tikzstyle{dots} = [decorate, draw]
\begin{frame}
	% \frametitle{Математическая логика и теория алгоритмов}
	\centering
	Документ для изучения предмета \\ \textbf{математическая логика и теория алгоритмов}
\end{frame}

\begin{frame}
	\frametitle{1. Доказать тавтологию}
	Формула $A$ называется \textbf{тавтологией} (или \textbf{тождественно истинной}), если формула истинна во всех интерпретациях. Значит формула \[f(x_1, x_2, \cdots, x_n) = 1 \text{, }\forall x_i \in \{0, 1\}\] , здесь (0 - ложность, 1 - истинность)
	
	\textbf{Нужно помнить}
	\begin{itemize}
		\item $A \rightarrow B$ \textbf{ложна} тогда и только тогда, когда $A=1$ и $B=0$ (так как $A \rightarrow B = \overline{A} \lor B)$
		\item $A \lor B$ \textbf{ложна} тогда и только тогда, когда $A=B=0$
		\item $A \land B$ \textbf{истинна} тогда и только тогда, когда $A=B=1$
	\end{itemize}
\end{frame}

\begin{frame}
	\frametitle{1. Доказать тавтологию}
	\textbf{Способ 1} (легко, но длинно): использовать таблицу истинности
	
	\textit{Пример}: $f(x, y) = [(x \Rightarrow y) \& (\lnot(\lnot y \Rightarrow \lnot x))] \Rightarrow \lnot x$ \\ Для краткости мы пишем $g = g(x, y) = x \Rightarrow y$ и $h = h(x, y) = \lnot (\lnot y \Rightarrow \lnot x)$. Тогда $f(x, y) = (g \& h) \Rightarrow \lnot x$
	\begin{table}
		\centering
		\begin{tabular}{|c|c|c|c|c|c|c|c|c|}
			\hline
			$x$ & $y$ & $g$ & $\lnot x$ & $\lnot y$ & $\lnot y \Rightarrow \lnot x$ & $h$ & $g \& h$ & $f(x, y)$ \\ \hline
			0 & 0 & 1 & 1 & 1 & 1 & 0 & 0 & 1 \\ \hline
			0 & 1 & 1 & 0 & 1 & 1 & 0 & 0 & 1 \\ \hline
			1 & 0 & 0 & 1 & 0 & 0 & 1 & 0 & 1 \\ \hline
			1 & 1 & 1 & 0 & 0 & 1 & 0 & 0 & 1 \\ \hline
		\end{tabular}
	\end{table}
	Отсюда $f(x, y) = 1$ для всех пар $(x, y)$, поэтому эта формула тавтология
\end{frame}

\begin{frame}
	\frametitle{1. Доказать тавтологию}
	\textbf{Способ 2} (меньше)
	
	\textit{Пример:} $f(x, y) = [(x \Rightarrow y) \& (\lnot(\lnot y \Rightarrow \lnot x))] \Rightarrow \lnot x$
	
	Мы предположим, что существуют пара $(x, y)$, которая делает функцию $f(x, y) = 0$. Согласно слайду 1, $f(x, y) = 0$ тогда и только тогда, когда 
	\begin{multicols}{2}
		$\begin{cases}
			[(x \Rightarrow y) \& (\lnot(\lnot y \Rightarrow \lnot x))] = 1\\ \lnot x = 0
		\end{cases}
		$
		
		$$\Rightarrow
		\begin{cases}
			(x \Rightarrow y) = 1 \\
			\lnot(\lnot y \Rightarrow \lnot x) = 1 \\
			x = 1
		\end{cases}
		$$
		Из $x = 1$ и $(x \Rightarrow y) = 1$, мы получим $y=1$. Поэтому \vfill\null\columnbreak $\lnot(\lnot y \Rightarrow \lnot x) = \lnot(0 \Rightarrow 0) = \lnot 1 = 0$  (неверно так как у нас уже есть $\lnot(\lnot y \Rightarrow \lnot x) = 1$)
		
		В заключение, не существует пара $(x,y)$, удовлетворяющая $f(x, y) = 0$. Значит $f(x,y)=1$ для всех $(x,y)$, тогда функция является тавтологией
	\end{multicols}
\end{frame}
\end{document}
