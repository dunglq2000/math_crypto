\section*{КМ 2 - 07/04/2021}
\subsection*{Вариант 1}
\begin{exercise}[1] Вероятность того, что радиоприемник проработает гарантийный срок без отказов равна 0,9. Какова вероятность того, что из данных пяти радиоприемников три проработают гарантийный срок без отказов?
	$n=5$, $p=0,9 \Rightarrow q = 1-p=0,1$ \\ Мы получим $P_5(3) = C^3_5 p^3 q^2 = C^3_5 \cdot (0,9)^3 \cdot (0,1)^2 = 0,0729$
\end{exercise}
\begin{exercise}[2] Студент знает 20 из 25 вопросов программы. Студенту предлагается три вопроса. Какова вероятность того, что студент знает только один из предложенных ему вопросов? Какова вероятность того, что студент знает ответы на все три предложенных ему вопроса?
	
	$| \Omega | = C^3_{25}$ \\ Пусть $A$ = \{студент знает только один из предложенных\} \\ Выбрать 1 вопрос, который студент знает $\Rightarrow C^1_{20}$ вариантов \\ Выбрать 2 вопрос, которые студент не знает $\Rightarrow C^2_5$ вариантов \\ $\Rightarrow P(A) = \frac{C^1_{20} C^2_5}{C^3_{25}} = \frac{2}{23}$ \\ $B$ = \{студент знает все 3 вопроса\} \\ $P(B) = \frac{C^3_{20}}{C^3_{25}} = \frac{57}{115}$
\end{exercise}

\begin{exercise}[3] В двух урнах находятся шары отличающиеся только цветом причем в первой урне 5 белых, 11 черных и 6 красных шаров, а во второй соответственно 10, 8, 6. Из каждой урны извлекли по одному шару. Какова вероятность того, что оба шара одного цвета?
	
	$|\Omega| = C^1_{5+11+6} C^1_{10 + 8 + 6} = C^1_{22} C^1_{24}$ \\ Пусть $A$ = \{оба шара одного цвета\}. У нас есть 3 случая
	\begin{itemize}
		\item белый: $C^1_5 C^1_{10}$ вариантов
		\item черный: $C^1_{11} C^1_8$ вариантов
		\item красный: $C^1_6 C^1_6$ вариантов
	\end{itemize}
	Ответ $P(A) = \frac{C^1_5 C^1_{10} + C^1_{11} C^1_8 + C^1_6 C^1_6}{C^1_{22} C^1_{24}} = \frac{29}{88}$
\end{exercise}

\begin{exercise}[4] В одном из двух ящиков лежит  3  белых и  7  черных шаров, а в другом лежит 5  белых  и   4   черных шара. Наугад выбирается один из ящиков и вынимается из него шар, который оказался белым. Найти вероятность того, что следующий извлеченный из этого же ящика шар тоже белый.

Пусть $A_1$ = \{первый ящик выбирается\} \\ $A_2$ = \{второй ящик выбирается\} \\ $A$ = \{первый шар - белый\} \\ $\Rightarrow P(A) = P(A_1) P(A | A_1) + P(A_2) P(A | A_2) = \frac{1}{2} \cdot \frac{3}{10} + \frac{1}{2} \cdot \frac{5}{9} = \frac{77}{180}$ \\ Отсюда, если пусть $B_1$ = \{первый белый шар из первого ящика\} \\ $B_2$ = \{первый белый шар из второго ящика\} \\ $B$ = \{второй шар - белый\} \\ мы получим
\begin{align*}
	P(B_1) & = P(A_1 | A) = \frac{1/2 \cdot 3/10}{77/180} = \frac{27}{77} \\ P(B_2) & = P(A_2 | A) = \frac{1/2 \cdot 5/9}{77/180} = \frac{50}{77}
\end{align*}
Если первый ящик выбирается, после мы взяли один белый шар, еще 8 белых, тогда $P(B | B_1) = \frac{2}{9}$. Аналогично, $P(B | B_2) = \frac{4}{8}$ \\ $\Rightarrow P(B) = P(B_1) P(B | B_1) + P(B_2) P(B | B_2) = \frac{27}{77} \cdot \frac{2}{9} + \frac{50}{77} \cdot \frac{4}{8} = \frac{31}{77}$
\end{exercise}

\begin{exercise}[5] Имеется  5 ящиков, в каждом из которых   10  белых  и  30  черных шаров. Из первого ящика во второй перекладывается один шар, затем из второго в третий перекладывается один шар  и т.д. После чего из последнего ящика извлекается один шар. Найти вероятность того, что он белый.

Пусть $A_1$ = \{белый шар из 1-ого ящика\} $\Rightarrow \overline{A_1}$ = \{черный шар из 1-ого ящика\} \\ $A_2$ = \{белый шар из 2-ого ящика\} $\Rightarrow \overline{A_2}$ = \{черный шар из 2-ого ящика\} \\ $A_3$ = \{белый шар из 3-ого ящика\} $\Rightarrow \overline{A_3}$ = \{черный шар из 3-ого ящика\} \\ $A_4$ = \{белый шар из 4-ого ящика\} $\Rightarrow \overline{A_4}$ = \{черный шар из 4-ого ящика\} \\ $A_5$ = \{белый шар из 5-ого ящика\} $\Rightarrow \overline{A_5}$ = \{черный шар из 5-ого ящика\}

Если мы выбираем белый шар из $i$-ого ящика, тогда в $i+1$-ом ящике есть 8+1=9 белых ящиков. Значит $P(A_{i+1} | A_i) = \frac{11}{41}$. Аналогично, если мы выбираем черный шар из $i$-ого ящика, $P(A_{i+1} | \overline{A_i}) = \frac{10}{41}$ и $$P(A_{i+1}) = P(A_i) P(A_{i+1} | A_i) + P(\overline{A_i}) P(A_{i+1} | \overline{A_i})$$ 

$P(A_1) = \frac{10}{40} = \frac{1}{4}$, $P(\overline{A_1}) = \frac{30}{40} = \frac{3}{4}$ \\ $\Rightarrow P(A_2) = P(A_1) P(A_2 | A_1) + P(\overline{A_1}) P(A_2 | \overline{A_1}) = \frac{1}{4} \cdot \frac{11}{41} + \frac{3}{4} \cdot \frac{10}{41} = \frac{1}{4}$ \\ $\Rightarrow P(\overline{A_2}) = 1-1/4=3/4$ \\ Аналогично, $P(A_2) = P(A_3) = P(A_4) = P(A_5) = \frac{1}{4}$ \\ Ответ $\frac{1}{4}$
\end{exercise}

\subsection*{Вариант 22}
\begin{exercise}[1] Из колоды карт (36 штук) выбирают наугад две карты. Какова вероятность того, что обе они окажутся пиковой масти?
	
	У нас есть 36 карт и выбирать 2 карты $\Rightarrow | \Omega | = C^2_{36}$ \\ Пусть $A$ = \{обе они окажутся пиковой масти\} \\ В колоде карт есть 9 пиковых мастей $\Rightarrow | \Omega_A | = C^2_9$ \\ $\Rightarrow P(A) = \frac{C^2_9}{C^2_{36}} = \frac{2}{35}$
\end{exercise}

\begin{exercise}[2] Из урны, содержащей пять белых и четыре черных шара, извлекают шары по одному без возвращения, пока не будет извлечен белый шар. Какова вероятность того, что придется извлечь четыре шара?
	
	Пусть $A_i$ = \{выбрать $i$ черных шаров, пока не будет извлечен белый шар\} $\Rightarrow i \in \{0, 1, 2, 3, 4\}$ \\ $B$ = \{выбрать белый шар в последний раз\} \\ У нас есть $P(B|A_0) = P(B|A_1) = \cdots = P(B|A_4) = \frac{1}{5}$
	\begin{align*}
		P(A_0) & = \frac{A^0_4}{A^0_9} = 1 \\ P(A_1) & = \frac{A^1_4}{A^1_9} = \frac{4}{9} \\ P(A_2) & = \frac{A^2_4}{A^2_9} = \frac{1}{6} \\ P(A_3) & = \frac{A^3_4}{A^3_9} = \frac{1}{21} \\ P(A_4) & = \frac{A^4_4}{A^4_9} = \frac{1}{126}
	\end{align*}
	\begin{align*}
		\Rightarrow P(A_3 | B) & = \frac{P(B|A_3)P(A_3)}{\sum_{i=0}^{4}P(B|A_i) P(A_i)} \\ & = \frac{1/5 \cdot 1/21}{1/5 \cdot (1 + 4/9 + 1/6 + 1/21 + 1/126)} = \frac{1}{35}
	\end{align*}
\end{exercise}

\begin{exercise}[3] Для каждого из чисел $X$ и $Y$ равновозможно любое значение из отрезка [0;2]. Какова вероятность того, что разность $X - Y$ превосходит единицу?
	
	У нас есть $X-Y \geq 1$ и $X, Y \in [0,2]$. Значит
	$\begin{cases}
		X - Y \geq 1 \\ 0 \leq X, Y \leq 2
	\end{cases}$
	Отсюда мы получим вероятность равно $P = \frac{S_{\triangle AB}}{S_{OBMN}} = \frac{1/2}{4} = \frac{1}{8}$
\end{exercise}

\begin{exercise}[4] В одном из двух ящиков лежит 9 белых и 3 черных шара, а в другом лежит 6 белых и 3 черных шара. Наугад выбирается один из ящиков и вынимается из него шар, который оказался белым. Найти вероятность того, что следующий извлеченный из этого же ящика шар тоже белый.
	
	Пусть $A_1$ = \{первый ящик выбирается\} \\ $A_2$ = \{второй ящик выбирается\} \\ $A$ = \{первый шар - белый\} \\ $\Rightarrow P(A) = P(A_1) P(A | A_1) + P(A_2) P(A | A_2) = \frac{1}{2} \cdot \frac{9}{12} + \frac{1}{2} \cdot \frac{6}{9} = \frac{17}{24}$ \\ Отсюда, если пусть $B_1$ = \{первый белый шар из первого ящика\} \\ $B_2$ = \{первый белый шар из второго ящика\} \\ $B$ = \{второй шар - белый\} \\ мы получим
	\begin{align*}
		P(B_1) & = P(A_1 | A) = \frac{1/2 \cdot 9/12}{17/24} = \frac{9}{17} \\ P(B_2) & = P(A_2 | A) = \frac{1/2 \cdot 6/9}{17/24} = \frac{8}{17}
	\end{align*}
	Если первый ящик выбирается, после мы взяли один белый шар, еще 8 белых, тогда $P(B | B_1) = \frac{8}{11}$. Аналогично, $P(B | B_2) = \frac{5}{8}$ \\ $\Rightarrow P(B) = P(B_1) P(B | B_1) + P(B_2) P(B | B_2) = \frac{9}{17} \cdot \frac{8}{11} + \frac{8}{17} \cdot \frac{5}{8} = \frac{127}{187}$
\end{exercise}

\begin{exercise}[5] Имеется  5 ящиков, в каждом из которых   8  белых  и  16  черных шаров. Из первого ящика во второй перекладывается один шар, затем из второго в третий перекладывается один шар  и т.д. После чего из последнего ящика извлекается один шар. Найти вероятность того, что он белый.
	Пусть $A_1$ = \{белый шар из 1-ого ящика\} $\Rightarrow \overline{A_1}$ = \{черный шар из 1-ого ящика\} \\ $A_2$ = \{белый шар из 2-ого ящика\} $\Rightarrow \overline{A_2}$ = \{черный шар из 2-ого ящика\} \\ $A_3$ = \{белый шар из 3-ого ящика\} $\Rightarrow \overline{A_3}$ = \{черный шар из 3-ого ящика\} \\ $A_4$ = \{белый шар из 4-ого ящика\} $\Rightarrow \overline{A_4}$ = \{черный шар из 4-ого ящика\} \\ $A_5$ = \{белый шар из 5-ого ящика\} $\Rightarrow \overline{A_5}$ = \{черный шар из 5-ого ящика\}
	
	Если мы выбираем белый шар из $i$-ого ящика, тогда в $i+1$-ом ящике есть 8+1=9 белых ящиков. Значит $P(A_{i+1} | A_i) = \frac{9}{25}$. Аналогично, если мы выбираем черный шар из $i$-ого ящика, $P(A_{i+1} | \overline{A_i}) = \frac{8}{25}$ и $$P(A_{i+1}) = P(A_i) P(A_{i+1} | A_i) + P(\overline{A_i}) P(A_{i+1} | \overline{A_i})$$ 
	
	$P(A_1) = \frac{8}{24} = \frac{1}{3}$, $P(\overline{A_1}) = \frac{16}{24} = \frac{2}{3}$ \\ $\Rightarrow P(A_2) = P(A_1) P(A_2 | A_1) + P(\overline{A_1}) P(A_2 | \overline{A_1}) = \frac{1}{3} \cdot \frac{9}{25} + \frac{2}{3} \cdot \cdot \frac{8}{25} = \frac{1}{3}$ \\ $\Rightarrow P(\overline{A_2}) = 1-1/3=2/3$ \\ Аналогично, $P(A_2) = P(A_3) = P(A_4) = P(A_5) = \frac{1}{3}$ \\ Ответ $\frac{1}{3}$
\end{exercise}