\section*{Занятие 3}
\begin{exercise}[1]
	
	\begin{enumerate}
		\item [(a)] Пусть $A$ - вероятность безотказной работы
		
		\begin{circuitikz}
			\draw
			(0,1) to[] (1,1)
			(1,1) to[] (1,0)
			(1,1) to[] (1,2)
			(1,2) to[generic,l=$1$] (3,2)
			(1,0) to[generic,l=$2$] (3,0)
			(3,1) to[] (3,2)
			(3,1) to[] (3,0)
			(3,2) to[generic,l=$3$] (5,2)
			(3,0) to[generic,l=$4$] (5,0)
			(5,1) to[] (5,0)
			(5,1) to[] (5,2)
			(5,2) to[generic,l=$5$] (7,2)
			(5,0) to[generic,l=$6$] (7,0)
			(7,1) to[] (7,0)
			(7,1) to[] (7,2)
			(7,2) to[generic,l=$7$] (9,2)
			(7,0) to[generic,l=$8$] (9,0)
			(9,1) to[] (9,0)
			(9,1) to[] (9,2)
			(9,1) to[] (10,1)
			;
		\end{circuitikz}
		
		\begin{align*}
			\Rightarrow P(A) & = P(A_1 + A_2) P(A_3 + A_4) P(A_5 + A_6) + P(A_7 + A_8) \\ & = [1-(1-P(A_1))(1-P(A_2))] \cdot [1-(1-P(A_3))(1-P(A_4))] \\ & \times [1-(1-P(A_5))(1-P(A_6))] \cdot [1-(1-P(A_7))(1-P(A_8))] \\ & = (1-(1-0,8)(1-0,8))^4 \\ & \approx 0,85
		\end{align*}
	
		\item [(б)] Пусть $A$ - вероятность безотказной работы
		
		\begin{circuitikz}
			\draw
			(0,1) to[] (1,1)
			(1,1) to[] (1,0)
			(1,1) to[] (1,2)
			(1,2) to[generic,l=$1$] (3,2)
			(1,0) to[generic,l=$2$] (3,0)
			(3,1) to[] (3,2)
			(3,1) to[] (3,0)
			(3,1) to[generic,l=$3$] (5,1)
			(5,1) to[] (5,0)
			(5,1) to[] (5,2)
			(5,2) to[generic,l=$4$] (7,2)
			(5,0) to[generic,l=$5$] (7,0)
			(7,1) to[] (7,0)
			(7,1) to[] (7,2)
			(7,1) to[] (8,1)
			;
		\end{circuitikz}
		
		\begin{align*}
			P(A) & = P(A_1 + A_2) P(A_3) P(A_4 + A_5) \\ & = [1 - (1-P(A_1))(1-P(A_2))] \cdot P(A_3) \cdot [1-(1-P(A_4))(1-P(A_5))] \\ & = (1-(1-0,8)^2) \cdot 0,8 \cdot (1-(1-0,8)^2) \\ & = 0,73728 \approx 0,74
		\end{align*}
	
		\item [(в)] Пусть $A$ - вероятность безотказной работы
		
		\begin{circuitikz}
			\draw
			(0,2) to[generic,l=$1$] (5,2)
			(5,2) to[generic,l=$2$] (10,2)
			(1,2) to[] (1,0)
			(1,0) to[] (3,0)
			(3,0) to[] (3,1)
			(3,0) to[] (3,-1)
			(3,1) to[generic,l=$3$] (7,1)
			(3,-1) to[generic,l=$4$] (7,-1)
			(7,0) to[] (7,1)
			(7,0) to[] (7,-1)
			(7,0) to[] (9,0)
			(9,0) to[] (9,2)
			;
		\end{circuitikz}
	
		\begin{align*}
			P(A) & = P(A_1 A_2 + A_3 + A_4) \\ & = 1 - (1 - P(A_1 A_2) (1 - P(A_3)) (1 - P(A_4))) \\ & = 1 - (1 - P(A_1) P(A_2) (1 - P(A_3)) (1 - P(A_4))) \\ & = 1 - (1 - (0,8)^2 \cdot (1 - 0,8) \cdot (1 - 0,8)) \\ & = 0,9856 \approx 0,98
		\end{align*}
		
	\end{enumerate}
\end{exercise}

\begin{exercise}[2]
	Пусть $A$ - событие после 4 шара появится черный шар \\ $A_1, A_2, A_3$ - события выбрать белый шар в \textit{i-ый} раз \\ $A_4$ - событие выбрать черный шар в последний раз \\ $P(A) = P(A_1) P(A_2) P(A_3) P(A_4)$
	
	\begin{enumerate}	
		\item [(a)] $P(A_1) = P(A_2) = P(A_3) = \frac{7}{10}$ \\ $P(A_4) = \frac{3}{10}$ \\ $\Rightarrow P(A) = \Big(\frac{7}{10}\Big)^3 \cdot \frac{3}{10} = \frac{1029}{10000} = 0,1029$
		\item [(б)] $P(A_1) = \frac{7}{10}$, $P(A_2) = \frac{6}{9}$, $P(A_3) = \frac{5}{8}$ \\ $P(A_4) = \frac{3}{7}$ \\ $\Rightarrow P(A) = \frac{7}{10} \cdot \frac{6}{9} \cdot \frac{5}{8} \cdot \frac{3}{7} = \frac{1}{8}$
	\end{enumerate}
\end{exercise}

\begin{exercise}[3]
	Пусть $A$ - событие среди них окажется по меньше мере одна кость с шестью очками \\ Тогда, $\bar{A}$ - событие нет кости с шестью очками. То $\bar{A}$ имеет 28-7=21 вариантов \\ $| \Omega | = C^7_{28}$ \\ $\Rightarrow P(A) = 1-P(\bar{A}) = 1 - \frac{C^7_{21}}{C^7_{28}} = \frac{2966}{3289} \approx 0,9$
\end{exercise}

\begin{exercise}[4]
	Пусть $A$ - событие на них выпадут разные грани \\ $| \Omega | = 6^4$ \\ Первая кость имеет 6 вариантов, вторая имеет 5, третья имеет 4 и четвертая имеет 3 \\ $P(A) = \frac{6 \cdot 5 \cdot 4 \cdot 3}{6^4} = \frac{5}{18}$
\end{exercise}

\begin{exercise}[5]
	Пусть $A$ - событие выбрать 2 шара одного цвета \\ $| \Omega | = C^2_{5+7+8} = C^2_{20}$ \\ Случай 1: выбрать 2 белого шара: $C^2_5$ \\ Случай 2: выбрать 2 красного шара: $C^2_7$ \\ Случай 3: выбрать 2 синего шара: $C^2_8$ \\ $\Rightarrow P(A) = \frac{C^2_5 + C^2_7 + C^2_8}{C^2_{20}} = \frac{59}{190} \approx \frac{1}{3}$ 
\end{exercise}
	
\begin{exercise}[6]
	Пусть $A$ - событие дуэль закончится гибелью одного из дуэлянтов. Значит один дуэль стреляет точно, а другой нет
	
	Выбрать человек, который стреляет точно: 2 \\ $\Rightarrow P(A) = 2 \cdot 0,2 \cdot (1 - 0,2) = 0,32$	
\end{exercise}

\begin{exercise}[7]
	Пространство элементарных событий: \\ выбрать 5 команд в первую группу: $C^5_{20}$ \\ Аналогично, выбрать 5 команд в группы 2, 3, 4: $C^5_{15}$, $C^5_{10}$, $C^5_5$ \\ $\Rightarrow | \Omega | = C^5_{20} C^5_{15} C^5_{10} C^5_5$
	
	Пусть $A$ - вероятность того, что в каждую подгруппу попадет по одному призеру \\ $\Rightarrow$ выбрать  подгруппы для 4 призера: $4!$ вариантов \\ Выбрать 4 команд в первую подгруппу: $C^4_{16}$ \\ Аналогично, выбрать 4 команд в подгруппы 2, 3, 4: $C^4_{12}$, $C^4_8$, $C^4_4$ \\ $\Rightarrow P(A) = \frac{4! C^4_{16} C^4_{12} C^4_8 C^4_4}{C^5_{20} C^5_{15} C^5_{10} C^5_5} = \frac{125}{969}$
	
	Пусть $B$ - событие 
\end{exercise}
	
\begin{exercise}[8]
	Пусть $D_i$ - событие выбрать один белый шар из i-ой урны \\ $\Rightarrow \bar{D_i}$ - событие не выбрать один белый шар из i-ой урны \\ Поэтому, $P(D_1) = \frac{2}{5}$, $P(D_2) = \frac{1}{3}$, $P(D_3) = \frac{3}{4}$
	
	\begin{enumerate}
		\item $A$ = \{вынуть только один белый шар\} \\ $\Rightarrow$ выбрать белый шар из 1-ой урны, или 2-ой, или 3-ей \begin{align*}
			\Rightarrow P(A) & = P(D_1 \cdot \overline{D_2} \cdot \overline{D_3} + \overline{D_1} \cdot D_2 \cdot \overline{D_3} + \overline{D_1} \cdot \overline{D_2} \cdot D_3) \\ & = P(D_1)[1-P(D_2)][1-P(D_3)] + [1-P(D_1)] P(D_2) [1-P(D_3)] + [1-P(D_1)] [1-P(D_2)] P(D_3) \\ & = \frac{2}{5} \cdot \Big(1 - \frac{1}{3}\Big) \cdot \Big(1 - \frac{3}{4}\Big) + \Big(1- \frac{2}{5}\Big) \cdot \frac{1}{3} \cdot \Big(1 - \frac{3}{4}\Big) + \Big(1 - \frac{2}{5}\Big) \cdot \Big(1 - \frac{1}{3}\Big) \cdot \frac{3}{4} = \frac{5}{12}
		\end{align*}
		\item $B$ - \{вынуть хотя бы один белый шар\} \\ $\Rightarrow \overline{B}$ - не вынуть ни одного белого шара. \\ $\Rightarrow \overline{B} = \overline{D_1} \cdot \overline{D_2} \cdot \overline{D_3}$ \\ $\Rightarrow P(\overline{B}) = P(\overline{D_1}) P(\overline{D_2}) \cdot P(\overline{D_3}) = \Big(1 - \frac{2}{5}\Big) \cdot \Big(1-\frac{1}{3}\Big) \cdot \Big(1 - \frac{3}{4}\Big) = \frac{1}{10}$ \\ $\Rightarrow P(B) = 1 - P(\overline{B}) = 1 - 0,1 = 0,9$
		\item $C$ - \{Вынуть шары различных цветов\} \\ Здесь у нас есть 4 случая:
		\begin{itemize}
			\item белый, синий, красный
			\item черный, белый, красный
			\item черный, синий, белый
			\item черный, синий, красный
		\end{itemize}
	Значит $C = A + \overline{B}$ \\ $\Rightarrow P(C) = P(A) + P(\overline{B}) = \frac{5}{12} + \frac{1}{10} = \frac{31}{60}$
	\end{enumerate}
\end{exercise}

\begin{exercise}
	$| \Omega | = C^2_{36}$ \\ Пусть $A$ - событие выбрать 2 красной масти \\ У нас есть итого 18 (36/2) красных мастей, поэтому $P(A) = \frac{C^2_{18}}{C^2_{36}} = \frac{17}{70}$
\end{exercise}