\section*{Занятие 13}

\begin{exercise}[1]
	$k=20$, $n=100$, $\gamma=0,90 \Rightarrow k/n = 1/5$ \\ У нас есть $2\Phi_0(1,65)=0,9 \Rightarrow t_\gamma=1,65$ \\ $\Rightarrow \frac{k}{n} - t_\gamma \sqrt{\frac{\frac{k}{n} \cdot \Big(1-\frac{k}{n}\Big)}{n}} < p < \frac{k}{n} + t_\gamma \sqrt{\frac{\frac{k}{n} \cdot \Big(1-\frac{k}{n}\Big)}{n}}$ \\ $\Rightarrow \frac{1}{5} - 1,65 \sqrt{\frac{1/5 \cdot 4/5}{100}} < p < \frac{1}{5} + 1,65 \sqrt{\frac{1/5 \cdot 4/5}{100}}$ \\ $\Rightarrow 0,134 < p < 0,266$
\end{exercise}

\begin{exercise}[2]
	$k=810$, $n=900$, $\gamma=0,95 \Rightarrow k/n = 9/10$ \\ $2\Phi_0(1,96)=0,95 \Rightarrow t_\gamma = 1,96$ \\ $\Rightarrow \frac{k}{n} - t_\gamma \sqrt{\frac{\frac{k}{n} \cdot \Big(1-\frac{k}{n}\Big)}{n}} < p < \frac{k}{n} + t_\gamma \sqrt{\frac{\frac{k}{n} \cdot \Big(1-\frac{k}{n}\Big)}{n}}$ \\ $\Rightarrow 0,8804 < p < 0,9196$
\end{exercise}

\begin{exercise}[3]
	$0,73 < p < 0,77$
\end{exercise}

\begin{exercise}[4]
	$k=1200$, $\overline{X} = 15,6$, $s=0,4$, $\gamma=0,95 \Rightarrow t_\gamma=1,96$ \\ $\overline{X} - t_\gamma \frac{s}{\sqrt{n}} < M < \overline{X} + t_\gamma \frac{s}{\sqrt{n}}$ \\ $\Rightarrow 15,3869 < M < 15,8131$
\end{exercise}

\begin{exercise}[5]
	По таблице 2, $t_\gamma = 1,96$, $8,363 < M(X) < 8,437$
\end{exercise}

\begin{exercise}[6]
	$\Rightarrow \frac{k}{n} - t_\gamma \sqrt{\frac{\frac{k}{n} \cdot \Big(1-\frac{k}{n}\Big)}{n}} < p < \frac{k}{n} + t_\gamma \sqrt{\frac{\frac{k}{n} \cdot \Big(1-\frac{k}{n}\Big)}{n}}$ \\ $\Rightarrow 0,26592 < p < 0,45408$
\end{exercise}