\section*{Занятие 4}
\begin{exercise}[1] 36 карт
	\begin{enumerate}
		\item [(a)] Пусть $A$ - событие выбрать 4 карты все разных мастей \\ $A_i$ - событие выбрать карту в $i$-ый раз	
		\begin{align*}
		P(A) = & P(A_1 \cdot A_2 \cdot A_3 \cdot A_4) \\ = & P(A_1) \cdot P(A_2|A_1) \cdot P(A_3|A_1 \cdot A_2) \\ & \times P(A_4 | A_1 \cdot A_2 \cdot A_3)
		\end{align*} 
	
		В первый раз, мы можем выбрать любую карту. Начало мы выбираем масть, потом число, значит $P(A_1) = \frac{4 \cdot 9}{36}$
		
		В второй раз, мы имеем 35 карт, выбрать масть и потом число, получим $P(A_2 | A_1) = \frac{3 \cdot 9}{35}$
		
		Далее, мы получим ответ $P(A) = \frac{4 \cdot 9}{36} \cdot \frac{3 \cdot 9}{34} \cdot \frac{2 \cdot 9}{34} \cdot \frac{1 \cdot 9}{33} = \frac{729}{6545}$
		\item [(б)] Пусть $B$ - событие выбрать 4 карты все разного достоинства \\ $B_i$ - событие выбрать карту в $i$-ый раз \begin{align*}
			P(B) &= P(B_1 \cdot B_2 \cdot B_3 \cdot B_4) \\ &= P(B_1) \cdot P(B_2|B_1) \cdot P(B_3|B_1 \cdot B_2) \cdot P(B_4 | B_1 \cdot B_2 \cdot B_3)
		\end{align*} 
	В первый раз, мы можем выбрать любую карту. Поэтому $P(B_1) = \frac{36}{36}$ \\ В второй раз, ещё 35 карт. Мы не можем выбрать карту с старого числом, значит мы имеем $36-4=32$ карта. $P(B_2 | B_1) = \frac{32}{35}$ \\ В третьи раз, ещё 34 карта. Мы не можем выбрать карту с старыми числами, значит $32-4=28$ карт. $P(B_3 | B_1 \cdot B_2) = \frac{28}{34}$ \\ Далее, мы получим ответ: $P(B) = \frac{36}{36} \cdot \frac{32}{35} \cdot \frac{28}{34} \cdot \frac{24}{33} = \frac{512}{935}$
	\end{enumerate}
\end{exercise}

\begin{exercise}[2]
	Пусть $A$ - событие в первом игре выбрать 2 мяча \\ $B$ - событие в втором игре выбрать 2 нового мяча \\ С событием $A$ у нас есть 3 варианта
	\begin{itemize}
		\item $A_1$ - два нового мяча. $P(A_1) = \frac{7 \cdot 6}{10 \cdot 9}$ \\ Тогда в втором игре есть 5 новых мячей, $P(B | A_1) = \frac{5 \cdot 4}{10 \cdot 9}$
		\item $A_2$ - один новый и одни побывавший. Мы можем выбрать новый мяч и потом побывавший, или обратно. $P(A_2) = \frac{2 \cdot 7 \cdot 3}{10 \cdot 9}$ \\ Тогда в втором игре есть 6 новых мячей, $P(B | A_2) = \frac{6 \cdot 5}{10 \cdot 9}$ 
		\item $A_3$ - 2 побывавшего мяча. $P(A_3) = \frac{3 \cdot 2}{10 \cdot 9}$ \\ Тогда в втором игре есть 7 новых мячей, $P(B | A_2) = \frac{7 \cdot 6}{10 \cdot 9}$
	\end{itemize}
	Ответ:
	\begin{align*}
		P(B) &= P(A_1) P(B | A_1) + P(A_2) P(B | A_2) + P(A_3) P(B | A_3) \\ &= \frac{7 \cdot 6 \cdot 5 \cdot 4 + 2 \cdot 7 \cdot 3 \cdot 6 \cdot 5 + 3 \cdot 2 \cdot 7 \cdot 6}{(10 \cdot 9)^2} = \frac{196}{675} \approx 0,29
	\end{align*}
\end{exercise}

\begin{exercise}[3]
	Пусть $A$ - событие благополучного полета \\ $B_i$ - событие на $i$-ом крыле сохраняет работоспособность \\ $\Rightarrow \overline{B_i}$ - событие на $i$-ом крыле 2 мотора не работают \\ $P(\overline{B_i}) = p^2$ \\ $\Rightarrow P(B_i) = 1-p^2$ \\ А у нас есть: $A = B_1 \cdot B_2$ \\ $\Rightarrow P(A) = P(B_1) \cdot P(B_2) = (1-p^2)^2$
\end{exercise}

\begin{exercise}[4]
	Пусть $A$ - \{студент сдаст экзамен\} \\ Пусть $A_1$ - \{правильно ответить 2 предложенных вопроса\} \\ Пусть $A_2$ - \{правильно ответить один из 2 предложенных вопроса и 1 дополнительный вопрос\} \\ У нас есть $P(A) = P(A_1) + P(A_2)$ \\ Для $A_1$: выбрать 2 предложенных вопроса, $P(A_1) = \frac{20 \cdot 19}{30 \cdot 29}$ \\ Для $A_2$: выбрать 2 предложенных вопроса: 20*19 вариантов, потом выбрать правильный вопрос: 2 варианта, и выбрать дополнительный вопрос: 10 вариантов \\ $\Rightarrow P(A_2) = \frac{20 \cdot 19 \cdot 2 \cdot 10}{30 \cdot 29 \cdot 28}$ \\ $\Rightarrow P(A) = \frac{20 \cdot 19}{30 \cdot 29} + \frac{20 \cdot 19 \cdot 2 \cdot 10}{30 \cdot 29 \cdot 28} = \frac{152}{203}$
\end{exercise}

\begin{exercise}[6]
	Пусть $X$ - количество бросков чтобы закончить
	\begin{enumerate}
		\item [(a)] (опят закончится до шестого броска) \\ Первый бросок, какая сторона не важно $\Rightarrow P(A) = P(X=2) + P(X=3) + P(X=4) + P(X=5)$ \\ Второй бросок может выпадет \textit{одной и той же} стороной, или \textit{другой}. Оба вероятности равны $\frac{1}{2}$ \\ То $P(X=2) = \frac{1}{2}$ \\ Если $P(X=3)$, третьи бросок будет другой стороной, чем 2 другие. Поэтому вероятность того, что второй бросок выпадает одной стороной: $\frac{1}{2}$, а третьи бросок: $\frac{1}{2}$ \\ То $P(X=3) = \frac{1}{2} \cdot \frac{1}{2} = \frac{1}{4}$, $P(X=4) = \frac{1}{8}$ и $\frac{1}{16}$ \\ Аналогично, $P(X=4) = \frac{1}{8}$ \\ $\Rightarrow P(A) = \frac{1}{2} + \frac{1}{4} + \frac{1}{8} + \frac{1}{16} = \frac{15}{16}$ 
		\item [(б)] $B$ - \{понадобится более четырех бросков\} \\ $\overline{B}$ - \{понадобится $\leq 4$\} \\ Делаем как (а), мы получим $P(\overline{B}) = \frac{1}{2} + \frac{1}{4} + \frac{1}{8} = \frac{7}{8}$ \\ $\Rightarrow P(B) = 1 - P(\overline{B}) = 1 - \frac{7}{8} = \frac{1}{8}$
	\end{enumerate}
\end{exercise}

\begin{exercise}[7] Из колоды карт (36 штук)
	\begin{enumerate}
		\item [(a)] Пусть $A$ = \{Первый тух появится при третьем извлечении карты\} \\ $| \Omega | = A^3_36$ \\ Пусть $a_1, a_2, a_3$ - карты выбраны \\ Тогда $a_3$ будет одном из 4 карт, у нас есть 4 вариантов \\ Ещё нужно выбрать карты $a_1$ и $a_2$, мы не можем выбрать туз, поэтому имеем $A^2_{36-4} = A^2_{32}$ \\ Ответ: $P(A) = \frac{4 \cdot A^2_{32}}{A^3_{36}} = \frac{496}{5355} \approx 0,09$
		\item [(б)] Пусть $B$ = \{Первый туз появится не ранее третьего извлечения карты\} \\ $\Rightarrow \overline{B}$ = \{Первый туз появится при первом или втором извлечении карты\}
		\begin{itemize}
			\item Если первый туз появится при первом извлечении, то вероятность равно $\frac{4}{36} = \frac{1}{9}$
			\item Если первый туз появится при втором извлечении, то делаем аналогично (а), мы получим $\frac{4 \cdot A^1_{32}}{A^2_{36}}$
		\end{itemize}
		Поэтому, $P(\overline{B}) = \frac{4}{36} + \frac{4 \cdot A^1_{32}}{A^2_{26}} = \frac{67}{315}$ \\ $\Rightarrow P(B) = 1 - P(\overline{B}) = \frac{248}{315}$
	\end{enumerate}
\end{exercise}

\begin{exercise}[8] Из колоды карт (36 карт) не более трех карт
	Пусть $A$ = \{выбрать более трех карт\} \\ $\overline{A}$ = \{выбрать не более трех карт, значит 1, 2 или 3 карт\}
	\begin{itemize}
		\item Если выбрать 1 карту, $| \Omega_1 | = 36$ \\ Выбрать одну красную карту из 18 красных карт, есть 18 вариантов. То $P(X=1) = \frac{18}{36}$ 
		\item Если выбрать 2 карты, $| \Omega_2 | = A^2_{36}$ \\ Выбрать одну красную карту в последнем месте, то есть 18 вариантов. А первое место мы не можем выбрать красную карту, то есть 36-18=18 вариантов. Поэтому $P(X=2) = \frac{18 \cdot 18}{A^2_{36}}$
		\item Если выбрать 3 карты, $| \Omega_3 | = A^3_{36}$ \\ Выбрать одну красную карту в последнем месте, есть 18 вариантов. Другие места мы не можем выбрать красные карты, а 2 черные карты из 18 карт. Поэтому $P(X=3) = \frac{18 \cdot A^2_{18}}{A^3_{36}}$
	\end{itemize}
	$\Rightarrow P(\overline{A}) = P(X=1) + P(X=2) + P(X=3) = \frac{18}{36} + \frac{18 \cdot 18}{A^2_{26}} + \frac{18 \cdot A^2_{18}}{A^3_{36}} = \frac{31}{35}$ \\ $\Rightarrow P(A) = 1 - P(\overline{A}) = \frac{4}{35}$
\end{exercise}

\begin{exercise}[9] 8 белых, 6 черных и 2 синих
	\begin{enumerate}
		\item [(a)] повторный выбор шаров, $| \Omega | = {16}^3$ \\ Выбрать первый шар, есть 8 вариантов (белых). Выбрать второй шар, есть 6 вариантов (черных). Выбрать третьи шар, есть 2 варианта (синих). Поэтому, вероятность равно $\frac{8 \cdot 6 \cdot 2}{{16}^3} = \frac{3}{128}$
		\item [(б)] бесповторный выбор шаров, $| \Omega | = A^3_{16}$ \\ Делаем аналогично (а), получим вероятность $\frac{8 \cdot 6 \cdot 2}{A^3_{16}} = \frac{1}{35}$
	\end{enumerate}
\end{exercise}

\begin{exercise}[10]
	Пусть $A$ = \{шар окажется белым\} \\ $B_1$ = \{Результат монеты - гебр\} \\ $B_2$ = \{Результат монеты - цифра\} \\ То $B_1$ и $B_2$ - полная группа событий, и $P(B_1) = P(B_2) = \frac{1}{2}$
	\begin{itemize}
		\item Если первая урна выбрана, то вероятность того, что шар окажется белым будет $P(A | B_1) = \frac{4}{4+2} = \frac{4}{6}$
		\item Если вторая урна выбрана, то вероятность того, что шар окажется белым будет $P(A | B_2) = \frac{3}{3 + 5} = \frac{3}{8}$
	\end{itemize}
	Ответ: $P(A) = P(B_1) \cdot P(A | B_1) + P(B_2) \cdot P(A | B_2) = \frac{1}{2} \cdot \frac{4}{6} + \frac{1}{2} \cdot \frac{3}{8} = \frac{25}{48}$
\end{exercise}