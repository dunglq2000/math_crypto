\section*{Занятия 6}

\begin{exercise}[1]
	Пусть $A$ = \{Два раза выпадет шесть очков\} \\ У нас есть $n = 5$, $p = \frac{1}{6} \Rightarrow q = 1-p = \frac{5}{6}$ \\ Ответ: $P(A) = P_5(k=2) = C^2_5 \cdot \Big(\frac{1}{6}\Big)^2 \cdot \Big(\frac{5}{6}\Big)^3 = \frac{625}{3888} \approx 0,16$
\end{exercise}

\begin{exercise}[2]
	У нас есть количество детей $n=4$, вероятность рождения мальчика $p=\frac{1}{2}$, поэтому вероятность рождения девочки $q = 1-p = \frac{1}{2}$ \\ Ответ $P_4(k=2) = C^2_4 \cdot \Big(\frac{1}{2}\Big)^2 \cdot \Big(\frac{1}{2}\Big)^2 = \frac{3}{8}$
\end{exercise}

\begin{exercise}[3]
	Пусть $A$ = \{первый стрелок попадает в цель дважды\} \\ $B$ = \{второй стрелок попадает в цель дважды\} \\ У нас есть $n_A = 4$, $p_A = \frac{1}{3} \Rightarrow q_A = 1 - p_A = \frac{2}{3}$ \\ $n_B = 3$, $p_B = \frac{1}{2} \Rightarrow q_B = 1-p_B = \frac{1}{2}$
	
	Поэтому, $P(A) = P_4(k=2) = C^2_4 \cdot \Big(\frac{1}{3}\Big)^2 \cdot \Big(\frac{2}{3}\Big)^2 = \frac{8}{27}$ 
	
	$P(B) = P_3(k=2) = C^2_3 \cdot \Big(\frac{1}{2}\Big)^2 \cdot \Big(\frac{1}{2}\Big) = \frac{3}{8}$
	Так как $P(B) > P(A)$, второго стрелка вероятнее
\end{exercise}

\begin{exercise}[4]
	Здесь у нас есть $n=5$, вероятность узла $p = 1 - 0,9 = 0,1$, и вероятность надежности $q = 0,9$
	\begin{enumerate}
		\item [(a)] $P_5(k=1) = C^1_5 \cdot (0,1)^1 \cdot (0,9)^4 = 0,32805$
		\item [(б)] $P_5(k \geq 1) = 1 - P_5(k=0) = 1 - C^0_5 \cdot (0,9)^5 \approx 0,4$
		\item [(в)] $P_5(k \geq 2) = 1 - P_5(k=0) - P_5(k=1) = 1 - C^0_5 \cdot 0,9^5 - C^1_5 \cdot 0,1 \cdot 0,9^4 = 0,08146 \approx 0,08$
	\end{enumerate}
\end{exercise}

\begin{exercise}[5]
	У нас есть $n = 100$ и $p=0,02$ - мало, поэтому мы можем использовать формулу Пуассона с $\lambda = np = 100 \cdot 0,02 = 2$
	\begin{enumerate}
		\item [(a)] $P_{100} (k=0) = \frac{e^{-2} \cdot 2^0}{0!} = e^{-2}$
		\item [(б)] $P_{100} (k>2) = 1-P_{100} (k = 0) - P_{100} (k = 1) = 1 - \frac{e^{-2} \cdot 2^0}{0!} - \frac{e^{-2} \cdot 2^1}{1!} - \frac{e^{-2} \cdot 2^2}{2!} = 1 - 5e^{-2}$
		\item [(в)] $P_{100} (k = k_0)$, мы используем $k=0, 1, 2, \cdots$ и видим, что $k=2$ наиболее. Поэтому ответ $k_0 = 2$
	\end{enumerate}
\end{exercise}

\begin{exercise}[6]
	Пусть $n$ - количество изделий, чтобы с вероятностью не менее 0,95 обнаружить хотя бы одно изделие низкого качества
	
	$\lambda = np = 0,1 n$, мы хотем, чтобы $P_n{k \geq 1} \geq 0,95$ \\ У нас есть
	\begin{align*}
		P_n(k \geq 1) & = 1 - P_n(k = 0) = 1 - \frac{e^{-\lambda} \cdot \lambda^0}{0!} = e^{-\lambda} \geq 0,95 \\ & \Leftrightarrow e^{-\lambda} \leq 0,05 \\ & \Leftrightarrow -\lambda \leq \ln{0,05} = -2,9957 \\ & \Leftrightarrow \lambda \geq 2,9957 \\ & \Leftrightarrow n \cdot 0,1 \geq 2,9957 \\ & \Leftarrow n \geq 29,957 \\ & \Leftrightarrow n \geq 30
	\end{align*}
\end{exercise}

\begin{exercise}[7]
	В течение часа поступает в среднем 120 телефонных вызовов, значит в течение минуты $\lambda = \frac{1 \cdot 120}{60} = 2$ вызовов.
	
	Поэтому ответ: $P(k=3) = \frac{e^{-2} \cdot 2^3}{3!} = \frac{2e^{-2}}{3} \approx 0,09$
\end{exercise}

\begin{exercise}[8]
	У нас есть $n=400$ и $p=0,005$ - мало, поэтому мы можем используем формулу Пуассона с $\lambda = np = 2$ \\ Получим
	\begin{align*}
		P_{400} (k > 2) & = 1 - P_{400}(k=0) - P_{400}(k=1) - P_{400}(k=2) \\ & = 1 - \frac{e^{-2} \cdot 2^0}{0!} - \frac{e^{-2} \cdot 2^1}{1!} - \frac{e^{-2} \cdot 2^2}{2!} \\ & = 1 - 5e^{-2} \approx 0,31
	\end{align*}
\end{exercise}

\begin{exercise}[9]
	У нас есть $n=800$ и $p=0,0025$ - вероятность цифра может быть принята неправильно \\ Чтобы получить текст, все цифры будут приняты правильно, $k=0$ и $\lambda = np = 800 \cdot 0,0025 = 2$ \\ Ответ: $P_{800}(k=0) = \frac{e^{-2} \cdot 2^0}{0!} = e^{-2} \approx 0,136$
\end{exercise}

\begin{exercise}[10]
	Мы знаем только один выстрел из 200 достигает цели, тогда при 100 выстрелах $\lambda = 1 \cdot 100 : 200 = \frac{1}{2}$ \\ Ответ $P_{100}(k \geq 1) = 1 - P_{100} (k = 0) = 1 - \frac{e^{\frac{-1}{2}} \cdot \Big(\frac{1}{2}\Big)^0}{0!} = 1 - e^{\frac{-1}{2} \approx 0,4}$
\end{exercise}

\begin{exercise}[11]
	У нас есть $n=50$ и $p=0,01$ - мало, поэтому мы используем формулу Пуассона с $\lambda = np = 0,5$ \\ Ответ $P_{50} (k \geq 1) = 1 - P_{50} (k = 0) = 1 - \frac{e^{-0,5} \cdot (0,5)^0}{0!} = 1 - e^{-0,5} \approx 0,4$
\end{exercise}

\begin{exercise}[12]
	Пусть $A_i$ = \{монету 1 бросают герб $i$ раз\}, и $B_i$ = \{монету 2 бросают герб $i$ раз\} \\ $C$ = \{у них выпадает одинаковое число гербов\}
	
	У нас есть $C = \sum_{i=0}^{5} A_i B_i$
	
	Мы образуем, что $P_{A, 5}(k=k_0)$ - вероятность того, что первая монета выпадает $k_0$ герб \\ $P_{B, 5}(k=k_0)$ - вероятность того, что вторая монета выпадает $k_0$ герб \\ Поэтому, 
	\begin{align*}
		P(C) & = \sum_{i=0}^{5} P_{A, 5}(k=i) \cdot P_{B, 5}(k=i) \\ & = \sum_{i=0}^{5} \Big[C^i_5 \Big(\frac{1}{2}\Big)^i \cdot \Big(\frac{1}{2}\Big)^{5-i}\Big] \cdot \Big[C^i_5 \Big(\frac{1}{2}\Big)^i \cdot \Big(\frac{1}{2}\Big)^{5-i}\Big] \\ & = \sum_{i=0}^{5} \Big[C^i_5 \Big(\frac{1}{2}\Big)^5\Big]^2 = \frac{63}{256}
	\end{align*}
\end{exercise}

\begin{exercise}[13]
	Пусть $x$ - количество раз частица двигается вправо и $y$ - количество раз частица двигается влево \\ Тогда, место частицы равна $x-y$ \\ Поэтому у нас есть 
	\begin{align*}
		x - y \in [-2; 0] \\ x + y = 5 \\ x, y \in \mathbb{Z}
	\end{align*}

	Среди \{-2, -1, 0\}, мы получим цельные числа только когда -1. Значит
	$\begin{cases}
		x - y = -1 \\ x + y = 5
	\end{cases}$

	Получим $x=2$, $y=3$. Пусть $A$ = \{через пять секунд частица двигается вправо 2 раза\} \\ Ответ $P_5 (k = 2) = C^2_5 \Big(\frac{1}{3}\Big)^2 \cdot \Big(\frac{2}{3}\Big)^3 = \frac{80}{243}$
\end{exercise}

\begin{exercise}[14] В биатлоне на каждом из трех огневых рубежей спортсмен должен поразить пять мишеней в пяти выстрелах. За каждую непораженную мишень спортсмен обязан пробежать штрафной круг. Пусть вероятность поражения мишени при одном выстреле равна 0,95. Какова вероятность того, что спортсмен все три огневых рубежа пройдет без штрафных кругов? какова вероятность того, что после каждого огневого рубежа спортсмен будет пробегать один штрафной круг?
	\begin{enumerate}
		\item [(a)] Пусть $A_i$ = \{спортсмен пройдет без штрафной круг на $i$-ом рубеже\}
		
		У нас есть $P(A_1) = P(A_2) = P(A_3)$
		\begin{align*}
			\Rightarrow P(A_1 A_2 A_3) & = P(A_1) P(A_2) P(A_3) = (P(A_1))^3 \\ & = (P_5(k=0))^3 = [C^0_5 (0,95)^5]^3 \\ & \approx 0,46
		\end{align*}
		\item [(б)] Пусть $B_i$ = \{спортсмен после $i$-ого огневого рубежа будет пробегать один штрафной круг\} \\ У нас есть $P(B_1) = P(B_2) = P(B_3)$ и $p=1-0,95=0,05$
		\begin{align*}
			\Rightarrow P(B_1 B_2 B_3) & = P(B_1) P(B_2) P(B_3) = (P(B_1))^3 \\ & = [P_5 (k=1)]^3 = [C^1_5 (0,05)^1 (0,95)^4]^3 \\ & \approx 0,008
		\end{align*}
	\end{enumerate}
\end{exercise}

\begin{exercise}[15]
	Пусть $A$ = \{к этому моменту у стрелка будет два промаха\}
	
	Мы знаем, что последний выстрел будет попадание. Среди первые 4 выстрела, там есть два промаха с вероятностью $p=1-\frac{1}{3} = \frac{2}{3}$ \\ Поэтому,
	\begin{align*}
		P(A) = \frac{1}{3} \cdot P_4(k=2) = \frac{1}{3} \cdot C^2_4 \cdot \Big(\frac{2}{3}\Big)^2 \cdot \Big(\frac{1}{3}\Big)^2 = \frac{8}{81}
	\end{align*}
\end{exercise}