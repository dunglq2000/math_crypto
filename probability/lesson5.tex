\section*{Занятия 5}

\begin{exercise}[1]
	Используем пример 5.1 и формулу Байеса:
	
	\[P(B_i | A) = \frac{P(B_i) \cdot P(A | B_i)}{P(A)} = \frac{P(B_i) \cdot P(A | B_i)}{\sum_{j=1}^{3} P(B_j) \cdot P(A | B_j)}\]
	\begin{itemize}
		\item $P(B_1) = \frac{\frac{1}{4} \cdot \frac{1}{4}}{\frac{17}{48}} = \frac{3}{17}$
		\item $P(B_2) = \frac{\frac{1}{4} \cdot \frac{1}{2}}{\frac{17}{48}} = \frac{6}{17}$
		\item $P(B_3) = \frac{\frac{1}{4} \cdot \frac{2}{3}}{\frac{17}{48}} = \frac{8}{17}$
	\end{itemize}
\end{exercise}

\begin{exercise}[3]
	Пусть $A_1$ = \{дефективные\} и $A_2$ = \{не дефективные\} \\ $B$ = \{Признан дефективным\} \\ Мы получим: $P(A_1) = 0,1$, $P(A_2) = 1-0,1=0,9$ \\ $P(B | A_1) = 0,95$, $P(B | A_2) = 0,03$ \\ $P(B) = P(A_1) P(B | A_1) + P(A_2) P(B | A_2) = 0,1 \cdot 0,95 + 0,9 \cdot 0,03 = 0,122$
\end{exercise}

\begin{exercise}[4]
	Пусть $A$ = \{событие болты производятся из А\} \\ $B$ = \{событие болты производятся из В\} \\ $C$ = \{событие болты производятся из С\} \\ $K$ = \{Болт оказался дефективным\} \\ У нас есть: $P(A) = 0,25$, $P(B) = 0,35$, $P(C) = 0,4$ \\ $P(K|A) = 0,05$, $P(K|B) = 0,04$, $P(K|C) = 0,02$ \\ Ответ: 
	\begin{align*}
		P(A|K) & = \frac{P(A) P(K|A)}{P(A)P(K|A)+P(B)P(K|B)+P(C)P(K|C)} \\ & = \frac{0,25 \cdot 0,05}{0,25 \cdot 0,05 + 0,35 \cdot 0,04 + 0,4 \cdot 0,02} = \frac{25}{69} \approx 0,36
	\end{align*}
\end{exercise}

\begin{exercise}[5]
	Пусть $A$ = \{вынуть белый шар из второй урны\} \\ $B$ = \{Вынуть два шара из первой урны\} \\ У нас есть $B = B_1 + B_2 + B_3$, с 
	\begin{itemize}
		\item $B_1$ = \{вынуть 1 белый и 1 черный\}. $P(B_1) = \frac{C^1_4 \cdot C^1_2}{C^2_6} = \frac{8}{15}$ \\ Теперь в второй урне есть 3 белых шара, поэтому $P(A | B_1) = \frac{3}{7}$
		\item $B_2$ = \{вынуть 2 белых\}. $P(B_2) = \frac{C^2_4}{C^2_6} = \frac{2}{5}$ \\ Теперь в второй урне есть 4 белых шара, поэтому $P(A | B_2) = \frac{4}{7}$
		\item $B_3$ = \{вынуть 2 черных\}. $P(B_3) = \frac{C^2_2}{C^2_6} = \frac{1}{15}$ \\ Теперь в второй урне есть 2 белых шара, поэтому $P(A | B_3) = \frac{2}{7}$
	\end{itemize}
	Ответ: 
	\begin{align*}
		P(A) = & P(B_1) \cdot P(A | B_1) + P(B_2) \cdot P(A | B_2) + P(B_3) \cdot P(A | B_3) \\ = & \frac{8}{15} \cdot \frac{3}{7} + \frac{2}{5} \cdot \frac{4}{7} + \frac{1}{15} \cdot \frac{2}{7} = \frac{10}{21}
	\end{align*}
\end{exercise}

\begin{exercise}[6]
	Пусть $A_1$ = \{выбрать 2 белых шара\}, $A_2$ = \{выбрать 2 черных шара\} и $A_3$ = \{выбрать 2 шара разного цвета\} \\ $B$ = \{третий шар отказался белым\} \\ У нас есть $P(A_3 | B) = \frac{P(A_3) \cdot P(B | A_3)}{P(A_1) + P(B|A_1) + P(A_2) \cdot P(B|A_2) + P(A_3) \cdot P(B | A_3)}$ (формула Байеса)
	\begin{itemize}
		\item C $A_1$, у нас есть $P(A_1) = \frac{5 \cdot 4}{8 \cdot 7} = \frac{5}{14}$ \\ Значит третий шар имеет 3 варианта, чтобы получить белый шар, $P(B | A_1) = \frac{3}{6} = \frac{1}{2}$
		\item C $A_2$, у нас есть $P(A_2) = \frac{3 \cdot 2}{8 \cdot 7} = \frac{3}{28}$ \\ Значит третий шар имеет 5 варианта, чтобы получить белый шар, $P(B | A_2) = \frac{5}{6}$
		\item C $A_3$, у нас есть $P(A_3) = \frac{2 \cdot 3 \cdot 5}{8 \cdot 7} = \frac{15}{28}$ (2 значит белый-черный или черный белый) \\ Значит третий шар имеет 4 варианта, чтобы получить белый шар, $P(B | A_3) = \frac{4}{6} = \frac{2}{3}$
	\end{itemize}
	Ответ: $P(A_3 | B) = \frac{15}{28} \cdot \frac{2}{3} : \Big(\frac{5}{14} \cdot \frac{1}{2} + \frac{3}{28} \cdot \frac{5}{6} + \frac{15}{28} \cdot \frac{2}{3}\Big) = \frac{4}{7}$
\end{exercise}

\begin{exercise}[7]
	Пусть $A_1$ = \{выбрать красный шар из первой урны\} и $A_2$ = \{выбрать черный шар из первой урны\} \\ $B_1$ = \{выбрать красный шар из второй урны\} и $B_2$ = \{выбрать черный шар из второй урны\} \\ У нас есть $P(A_1) = P(A_2) = \frac{1}{2}$ \\ $P(B_1 | A_1) = \frac{3}{6}$, $P(B_1 | A_2) = \frac{2}{6}$ \\ $P(B_2 | A_1) = \frac{3}{6}$, $P(B_2 | A_2) = \frac{4}{6}$
	\begin{enumerate}
		\item [(a)] Пусть $C$ = \{вынуты шары одного цвета\} 
		\begin{align*}
			\Rightarrow P(C) = & P(A_1 \cdot B_1 + A_2 \cdot B_2) \\ = & P(A_1) \cdot P(B_1 | A_1) + P(A_2) \cdot P(B_2 | A_2) \\ = & \frac{1}{2} \cdot \frac{3}{6} + \frac{1}{2} \cdot \frac{4}{6} = \frac{7}{12}
		\end{align*}
		\item [(б)] Пусть $D$ = \{выбрать красный шар из первой урны и черный шар из второй урны\}
		\begin{align*}
			P(D) = P(A_2 | B_1) = & \frac{P(A_2) \cdot P(B_1 | A_2)}{P(A_1) \cdot P(B_1 | A_1) + P(A_2) \cdot P(B_1 | A_2)} \\ = & \Big(\frac{1}{2} \cdot \frac{2}{6}\Big) : \Big(\frac{1}{2} \cdot \frac{2}{6} + \frac{1}{2} \cdot \frac{3}{6}\Big) = \frac{2}{5}
		\end{align*} 
	\end{enumerate}
\end{exercise}

\begin{exercise}[8]
	
\end{exercise}

\begin{exercise}[9]
	Пусть $A_1$ = \{первый шар является белым\} и $A_2$ = \{второй шар является черным\} \\ $B$ = \{два следующие шары - черные\} \\ У нас есть, $P(A_1) = \frac{4}{7}$, $P(A_2) = \frac{3}{7}$
	\begin{itemize}
		\item Если первый шар является белым, то в урне есть 3 белых и 3 черных, поэтому $P(B | A_1) = \frac{C^2_3}{C^2_6} = \frac{1}{5}$
		\item Если первый шар является черным, то в урне есть 4 белых и 2 черных, поэтому $P(B | A_2) = \frac{C^2_2}{C^2_6} = \frac{1}{15}$
	\end{itemize}
	Мы получим:
	\begin{align*}
		P(A_1 | B) & = \frac{P(A_1) \cdot P(B | A_1)}{P(A_1) \cdot P(B | A_1) + P(A_2) \cdot P(B | A_2)} \\ & = \Big(\frac{4}{7} \cdot \frac{1}{5}\Big) : \Big(\frac{4}{7} \cdot \frac{1}{5} + \frac{3}{7} \cdot \frac{1}{15}\Big) = \frac{4}{5} = 0,8
	\end{align*}
\end{exercise}

\begin{exercise}[10]
	Пусть $A_1$ = \{комбинация 11111 передана\} и $A_2$ = \{комбинация 00000 передана\} \\ $B$ = \{комбинация 10110 получена\} \\ У нас есть: $P(A_1) = 0,7$, $P(A_2) = 0,3$
	\begin{itemize}
		\item Если комбинация 11111 передана, то первое, третье и пятое цифры правильно, а другие нет, поэтому $P(B | A_1) = 0,6 \cdot 0,4 \cdot 0,6 \cdot 0,6 \cdot 0,4 = 0,6^3 \cdot 0,4^2$
		\item Если комбинация 00000 передана, то второе и четвертое цифры правильно, а другие нет, поэтому $P(B | A_2) = 0,4 \cdot 0,6 \cdot 0,4 \cdot 0,4 \cdot 0,6 = 0,4^3 \cdot 0,6^2$
	\end{itemize}
	Ответ
	\begin{align*}
		P(A_1 | B) & = \frac{P(A_1) \cdot P(B | A_1)}{P(A_1) \cdot P(B | A_1) + P(A_2) \cdot P(B | A_2)} \\ & = \frac{0,7 \cdot 0,6^3 \cdot 0,4^2}{0,7 \cdot 0,6^3 \cdot 0,4^2 + 0,3 \cdot 0,4^3 \cdot 0,6^2} = \frac{7}{9} \approx 0,78
	\end{align*}
\end{exercise}