\begin{frame}{Lý thuyết nhóm. Bài tập}
    \textbf{Bài tập 1 (giải thích)}
    \begin{enumerate}
        \item (Đã giải thích)
        \item Các số phức trong $S$ có dạng $z = a+bi$ với $a, b \in \mathbb{R}$ và $i^2=-1$. 
        \begin{itemize}
            \item Với 2 số phức $z_1 = a_1 + b_1 i$ và $z_2 = a_2 + b_2 i$, $z_1 z_2 = (a_1 a_2 - b_1 b_2) + (a_1 b_2 + a_2 b_1)i$. Do $|z_1| = \sqrt{a_1^2+b_1^2}=1$ và $|z_2| = \sqrt{a_2^2 + b_2^2} = 1$, ta có
            $|z_1 z_2| = \sqrt{(a_1 a_2 - b_1 b_2)^2 + (a_1 b_2 + a_2 b_1)^2} = \sqrt{(a_1 a_2)^2 - 2a_1a_2b_1b_2 + (b_1b_2)^2 + (a_1b_2)^2 + 2a_1b_2a_2b_1 + (a_2b_1)^2} = \sqrt{(a_1^2 + b_1^2)(a_2^2 + b_2^2)} = 1$. Như vậy $S$ có tính đóng
            \item Phần tử đơn vị: $e = 1 + 0i = 1$
            \item Phần tử nghịch đảo: $z' = a-bi$
            \item Tính kết hợp (như với phép nhân số phức thông thường)
        \end{itemize}
        \item Với $a, b \in (\mathbb{Z}/p\mathbb{Z})^{\times}$ thì $ab \pmod p$ cũng thuộc nhóm. Phần tử nghịch đảo tồn tại vì $\gcd(a, p) = 1$ với mọi $a \in (\mathbb{Z}/p\mathbb{Z})^{\times}$.
    \end{enumerate}
\end{frame}