\documentclass{beamer}
% \usepackage[utf8]{inputenc}
\usepackage[utf8]{vietnam}
\usepackage{amsmath}

\usetheme{Boadilla}
\usecolortheme{orchid}
\setbeamertemplate{footline}[frame number]
\beamertemplatenavigationsymbolsempty

\usepackage{tikz}
\tikzstyle{process} = [rectangle, minimum width=3cm, minimum height=1cm, text centered, draw=black, fill=orange!30]

\title{Tổng quan Lý thuyết nhóm}
\date{\today}

\begin{document}
\frame{\titlepage}
\section{Mở đầu}
\begin{frame}{Nội dung}
    \tableofcontents[
        currentsection,
        % currentsubsection,
        % hideothersubsections,
        sectionstyle=show/shaded,
        subsectionstyle=show/show/hide
    ]
\end{frame}
\begin{frame}{Lý thuyết tập hợp}
\textbf{Tập hợp (Set)}: một tập hợp bao gồm các phần tử, mỗi phần tử đều khác nhau. 

\pause
\textbf{Biểu diễn tập hợp}: có 2 cách biểu diễn tập hợp
\begin{enumerate}
    \item Liệt kê. Ví dụ $A = \{1, 2, 3, 6\}$
    \item Dùng tính chất đặc trưng của các phần tử. Ví dụ $A = \{x \in \mathbb{N} \text{  |  } 6 \vdots x\}$
\end{enumerate}

\pause
\textbf{Tập hợp rỗng (Empty set)}: tập hợp không có phần tử nào. Ký hiệu $\varnothing$

\pause
\textbf{Tập hợp con (Subset)}: tập hợp $B$ được gọi là tập hợp con của tập hợp $A$ nếu tất cả phần tử của $B$ đều thuộc $A$

Ví dụ: $B=\{1, 2, 3\}$, $A=\{1, 2, 3, 4\}$. Ta ký hiệu $B \subset A$

Như vậy tập hợp rỗng là tập hợp con của mọi tập hợp.

\pause
\textbf{Lực lượng của tập hợp}: là số phần tử của tập hợp. Ký hiệu là $|A|$. Ví dụ với $A=\{1,2,3,4\}$ thì $|A|=4$
\end{frame}

\begin{frame}{Lý thuyết số (Number theory)}
    \textbf{Chia hết (Divisibility)}. số nguyên $a$ \textit{chia hết cho/bị chia hết bởi} số nguyên $b$, nếu tồn tại số nguyên $c$ sao cho $a = b c$. Ký hiệu $a \vdots b$ hoặc $b | a$. Khi đó $b$ gọi là ước của $a$ và $a$ gọi là bội của $b$ \pause
    
    \textbf{Ước chung. Bội chung}. 
    \begin{itemize}
        \item Nếu số nguyên $d$ vừa là ước của $a$, vừa là ước của $b$ thì $d$ được gọi là ước chung của $a$ và $b$. Ví dụ: $\{1, 2\}$ là tập các ước chung của 2 và 4 \\ Trong tất cả ước chung ta quan tâm tới ước chung lớn nhất. 
        \item Nếu số nguyên $c$ vừa là bội của $a$, vừa là bội của $b$ thì $c$ được gọi là bội chung của $a$ và $b$. Ví dụ các số $\{15, 30, 45, \cdots \}$ là các bội chung của 3 và 5 \\ Trong tất cả bội chung ta quan tâm bội chung nhỏ nhất
    \end{itemize}
    \pause
    \textbf{Số nguyên tố (Prime number)}. Nếu số nguyên dương $p$ chỉ đúng 2 ước là 1 và chính nó thì $p$ gọi là số nguyên tố. Nếu không thì là \textbf{hợp số}. Ví dụ các số nguyên tố: \{2, 3, 5, 7, 11, 13, 17, 19, ....\}
\end{frame}

\section{Lý thuyết nhóm}
\begin{frame}{Nội dung}
    \tableofcontents[
        currentsection,
        % currentsubsection,
        % hideothersubsections,
        sectionstyle=show/shaded,
        subsectionstyle=show/show/hide
    ]
\end{frame}
\begin{frame}{Lý thuyết nhóm (Group theory)}
    \textbf{Định nghĩa.} \textit{Nhóm (Group)} $(G, \star)$ gồm một tập hợp $G$ các phần tử và phép toán 2 ngôi $\star$ trên các phần tử thuộc $G$.
    
    \pause
    Một nhóm thỏa mãn các tính chất sau:
    \begin{itemize}
        \item Tính đóng (Closure): với 2 phần tử $a, b$ bất kì thuộc $G$, $a\star b \in G$
        \item Phần tử đơn vị (Identity Law): tồn tại phần tử $e$ sao cho với mọi phần tử $a \in G$, $a \star e = e \star a = a$
        \item Phần tử nghịch đảo (Inverse Law): với mọi phần tử $a\in G$, tồn tại phần tử $a' \in G$ sao cho $a \star a' = a' \star a = e$
        \item Tính kết hợp (Associative Law): với mọi $a, b, c \in G$, $(a \star b) \star c = a \star (b \star c)$
    \end{itemize}
\end{frame}

\begin{frame}{Lý thuyết nhóm. Ví dụ}
    \textbf{Ví dụ 1.} Tập hợp số nguyên $\mathbb{Z}$ và $\star$ là phép cộng ($+$) thông thường trên tập số nguyên
    \pause
    \begin{itemize}
        \item Với $a, b \in \mathbb{Z}$, $a + b \in \mathbb{Z}$
        \item Phần tử đơn vị là 0. Với mọi $a \in \mathbb{Z}$ thì $a + 0 = 0 + a = a$
        \item Phần tử nghịch đảo của $a$ là $-a$, vì $a + (-a) = (-a) + a = 0$
        \item Tính kết hợp: $(a + b) + c = a + (b + c)$
    \end{itemize}
\end{frame}

\begin{frame}{Lý thuyết nhóm. Ví dụ}
    \textbf{Ví dụ 2.} Tập hợp số thực $\mathbb{R}\backslash \{0\}$ và $\star$ là phép nhân ($\times$) thông thường
    \pause
    \begin{itemize}
        \item Với $a, b \in \mathbb{R} \backslash \{0\}$, $a \times b \in \mathbb{R} \backslash \{0\}$
        \item Phần tử đơn vị là 1. Với mọi $a \in \mathbb{R} \backslash \{0\}, a \times 1 = 1 \times a = a$
        \item Phần tử nghịch đảo của $a$ là $\frac{1}{a}$, vì $a \times \frac{1}{a} = \frac{1}{a} \times a = 1$
        \item Tính kết hợp: $(a \times b) \times c = a \times (b \times c)$
    \end{itemize}
\end{frame}

\begin{frame}{Nhóm Abel. Nhóm vòng}
    Nếu nhóm có tính giao hoán (Commutative Law): $a \star b = b \star a$, $\forall a, b \in G$, thì nhóm được gọi là \textbf{nhóm Abel}
    \pause
    
    \textbf{Phép lũy thừa}: với nhóm $(G, \star)$ ta định nghĩa phép lũy thừa
    \begin{itemize}
        \item $g^k = \underbrace{g \star g \star \cdots \star g}_{k \text{ lần}}$ với $g$ là phần tử thuộc $G$ và $k \in \mathbb{N}$
        \item $g^{-k} = (g')^k$
        \item $g^0 = e$
    \end{itemize}
    \pause
    
     Ta gọi $G$ là \textbf{nhóm vòng (cyclic group)} nếu mọi phần tử trong $G$ đều có thể biểu diễn dưới dạng $g^k$ với $g \in G$ và  $k \in \mathbb{N}$.
     
     Khi đó $g$ được gọi là \textbf{phần tử sinh} (generator) của nhóm.
    \pause
    
    Ví dụ: nhóm $(\mathbb{Z}, +)$ với phần tử sinh là 1. Vì với mọi $a \in \mathbb{Z}^+$, ta có $a = \underbrace{1+1+\cdots+1}_{a \text{ lần}}$
\end{frame}

\begin{frame}{Lý thuyết nhóm. Bài tập}
    \textbf{Bài tập 1.} Các tập hợp và toán tử sau có tạo thành nhóm hay không?
    \begin{enumerate}
        \item $(\mathbb{Z}, \times)$
        \item $(S, \times)$. Với $S$ là tập hợp các số phức có mô đun là 1, tức là $S = \{z \in \mathbb{C} \text{ | } |z| = 1\}$
        \item $((\mathbb{Z}/p\mathbb{Z})^{\times}, \times)$. Với $(\mathbb{Z}/p\mathbb{Z})^{\times}$ là tập hợp các thặng dư khác 0 modulo $p$, trong đó $p$ là số nguyên tố. (tập hợp $\{1, 2, \cdots, p-1\}$)
    \end{enumerate}
    \pause
    \textbf{Bài tập 1 (Đáp án)}
    \begin{enumerate}
        \item Không phải vì phần tử đơn vị là 1 nhưng không tồn tại nghịch đảo với mọi $a \in \mathbb{Z}$
        \item $(S, \times)$ là một nhóm
        \item $((\mathbb{Z}/p\mathbb{Z})^{\times}, \times$ là nhóm
    \end{enumerate}
\end{frame}

\begin{frame}{Lý thuyết nhóm. Bài tập}
    \textbf{Bài tập 1 (giải thích)}
    \begin{enumerate}
        \item (Đã giải thích)
        \item Các số phức trong $S$ có dạng $z = a+bi$ với $a, b \in \mathbb{R}$ và $i^2=-1$. 
        \begin{itemize}
            \item Với 2 số phức $z_1 = a_1 + b_1 i$ và $z_2 = a_2 + b_2 i$, $z_1 z_2 = (a_1 a_2 - b_1 b_2) + (a_1 b_2 + a_2 b_1)i$. Do $|z_1| = \sqrt{a_1^2+b_1^2}=1$ và $|z_2| = \sqrt{a_2^2 + b_2^2} = 1$, ta có
            $|z_1 z_2| = \sqrt{(a_1 a_2 - b_1 b_2)^2 + (a_1 b_2 + a_2 b_1)^2} = \sqrt{(a_1 a_2)^2 - 2a_1a_2b_1b_2 + (b_1b_2)^2 + (a_1b_2)^2 + 2a_1b_2a_2b_1 + (a_2b_1)^2} = \sqrt{(a_1^2 + b_1^2)(a_2^2 + b_2^2)} = 1$. Như vậy $S$ có tính đóng
            \item Phần tử đơn vị: $e = 1 + 0i = 1$
            \item Phần tử nghịch đảo: $z' = a-bi$
            \item Tính kết hợp (như với phép nhân số phức thông thường)
        \end{itemize}
        \item Với $a, b \in (\mathbb{Z}/p\mathbb{Z})^{\times}$ thì $ab \pmod p$ cũng thuộc nhóm. Phần tử nghịch đảo tồn tại vì $\gcd(a, p) = 1$ với mọi $a \in (\mathbb{Z}/p\mathbb{Z})^{\times}$.
    \end{enumerate}
\end{frame}

\begin{frame}{Lý thuyết nhóm. Bài tập}
    \textbf{Bài tập 2.} Cho nhóm $S_3$ bao gồm các phần tử sau
    \begin{center}
        $e$, $\sigma$, $\sigma^2$, $\tau$, $\sigma\tau$, $\sigma^2\tau$
    \end{center}
    với $e$ là phần tử đơn vị và phép nhân được thực hiện theo quy tắc sau
    \begin{center}
        $\sigma^3 = e$, $\tau^2 = e$, $\tau\sigma = \sigma^2\tau$
    \end{center}
    Tính 
    \begin{enumerate}
        \item [(a)] $\tau\sigma^2$
        \item [(b)] $\tau(\sigma\tau)$
        \item [(c)] $(\sigma\tau)(\sigma\tau)$
        \item [(d)] $(\sigma\tau)(\sigma^2\tau)$
    \end{enumerate}
    Câu hỏi phụ: $S_3$ có tính giao hoán không? Vì sao?
\end{frame}

\begin{frame}{Lý thuyết nhóm. Bài tập}
    \textbf{Bài tập 2 (Đáp án)}
    \begin{enumerate}
        \item [(a)] $\tau\sigma^2 = (\tau\sigma)\sigma=(\sigma^2\tau)\sigma = (\sigma^2)(\tau\sigma)=(\sigma^2)(\sigma^2\tau)=\sigma^4\tau=\sigma\tau$
        \item [(b)] $\tau(\sigma\tau)=(\tau\sigma)\tau = (\sigma^2\tau)\tau=(\sigma^2)(\tau^2)=(\sigma^2)e=\sigma^2$
        \item [(c)] $(\sigma\tau)(\sigma\tau) = \sigma(\tau\sigma)\tau = \sigma(\sigma^2\tau)\tau=(\sigma^3)(\tau^2)=e e = e$
        \item [(d)] $(\sigma\tau)(\sigma^2\tau)=(\sigma\tau)(\tau\sigma)=\sigma(\tau^2)\sigma=\sigma e \sigma = \sigma \sigma = \sigma^2$
    \end{enumerate}
    Dễ thấy, $\tau\sigma = \sigma^2\tau \Rightarrow \tau\sigma \neq \sigma\tau$
    
    Do đó $S_3$ không giao hoán
    
    \textbf{Đọc thêm.} Nhóm như trên là nhóm Dihedral $D_{2n}$ (ở đây $n=3$)
\end{frame}

\begin{frame}{Nhóm con (Subgroup)}
    \textbf{Số lượng phần tử của nhóm $G$ (order)} được ký hiệu là $\# G$ \pause
    
    \textbf{Cấp của một phần tử (order)}: với phần tử $g \in G$, số nguyên dương $k$ nhỏ nhất để $g^k = e$ được gọi là \textit{cấp} của phần tử $g$. \pause
    
    \textbf{Tính chất}: Đặt $n = \# G$
    \begin{itemize}
        \item Với mọi $g \in G$, $g^n = e$
        \item Nếu phần tử $g \in G$ có order là $k$, khi đó $k | n$
    \end{itemize} \pause
    
    Nếu $k = n$, $g$ là phần tử sinh (generator) của nhóm $G$. \pause

    \textbf{Nhóm con} của nhóm $(G, \star)$ là nhóm $(G', \star)$, với $G' \subset G$. Khi đó,
    \begin{itemize}
        \item Với mọi $a, b \in G'$, $a \star b \in G'$
        \item Phần tử đơn vị là $e$ chính là phần tử đơn vị của $G$
        \item Phần tử nghịch đảo là $a' \in G'$
        \item Tính kết hợp: với mọi $a, b, c \in G'$, $(a \star b) \star c = a \star (b \star c)$
    \end{itemize}
\end{frame}

\begin{frame}{Nhóm con. RSA}
    \textit{Tại sao cần nhóm con?} Vì việc tính toán trên nhóm con thường nhanh hơn (nhiều lần) do miền giá trị nhỏ hơn. \pause
    
    Ví dụ: trong thuật toán RSA, quá trình giải mã $m = c^d \pmod N$, chúng ta thường chọn $e=65537$ nên $d$ tìm được sẽ rất lớn. Do đó việc giải mã sẽ rất chậm. 
    
    Do $N=pq$ với $p$ và $q$ là 2 số nguyên tố phân biệt, ta có thể dùng định lý số dư Trung Hoa để tăng tốc độ tính lên 4 lần.
\end{frame}

\begin{frame}{Nhóm con. RSA}
    Ngoài $p$ và $q$ để tính $N=pq$, ta cần thêm các tham số $dP$, $dQ$ và $qInv$ như sau:
    \begin{itemize}
        \item $dP.e = 1 \bmod p$
        \item $dQ.e = 1 \bmod q$
        \item $qInv.q = 1 \bmod p$ (với giả định $p > q$)
    \end{itemize}
    \pause
    Áp dụng định lý số dư Trung Hoa theo phương pháp Garner, thay vì tính $m = c^d \pmod N$ ta có thể tính $m$ như sau:
    \begin{itemize}
        \item $m_1 = c^{dP} \bmod p$
        \item $m_2 = c^{dQ} \bmod q$
        \item $h = qInv (m_1 - m_2) \pmod p$
        \item $m = m_2 + qh$
    \end{itemize}
    Với cách thực hiện như trên ta chỉ cần tính phép modulo $p$ và $q$ nhỏ hơn rất nhiều so với modulo $N$
\end{frame}

\section{Vành (sơ lược)}
\begin{frame}{Nội dung}
    \tableofcontents[
        currentsection,
        % currentsubsection,
        % hideothersubsections,
        sectionstyle=show/shaded,
        subsectionstyle=show/show/hide
    ]
\end{frame}
\begin{frame}{Vành}
    \textbf{Định nghĩa.} \textit{Vành (Ring)} $(R, +, \times)$ gồm tập hợp $R$ các phần tử và hai phép toán 2 ngôi là phép cộng ($+$) và phép nhân ($\times$) thỏa mãn các tính chất sau: \pause
    \begin{itemize}
        \item $R$ là nhóm Abel đối với phép cộng. Ta ký hiệu phần tử đơn vị của phép cộng là 0 và phần tử nghịch đảo của $a$ trong phép cộng là $-a$. Phép trừ $a - b = a + (-b)$
        \item Tính đóng đối với phép nhân: với 2 phần tử $a, b$ bất kì thuộc $R$, $a \times b \in R$
        \item Tính kết hợp đối với phép nhân: với mọi $a, b, c \in R$, $(a \times b) \times c = a \times (b \times c)$
        \item Tính phân phối giữa phép cộng và phép nhân: với mọi $a, b, c \in $
        \begin{align*}
            (a + b) \times c = a \times c + b \times c \\ a \times (b + c) = a \times b + a \times c
        \end{align*}
    \end{itemize}
\end{frame}

\begin{frame}{Vành}
    Khi vành có tính giao hoán đối với phép nhân, vành được gọi là \textbf{vành giao hoán}
    \begin{itemize}
        \item với mọi $a, b \in R$, $a \times b = b \times a$
    \end{itemize}
    \pause
    Một vành được gọi là \textbf{miền nguyên (integral domain)} nếu nó là vành giao hoán và có thêm 2 tính chất sau
    \begin{itemize}    
        \item Phần tử đơn vị đối với phép nhân, ký hiệu là 1: $1 \times a = a \times 1 = a$
        \item Liên quan giữa phép nhân và phần tử đơn vị của phép cộng: nếu $a \times b = 0$ thì $a = 0$ hoặc $b = 0$
    \end{itemize}
\end{frame}

\begin{frame}{Vành. Ví dụ}
    \textbf{Ví dụ 1.} Các tập $\mathbb{Z}$, $\mathbb{Q}$, $\mathbb{R}$ với phép cộng và nhân thông thường tạo thành vành. \pause

    \textbf{Ví dụ 2 (vành đa thức).} Tập hợp các đa thức $$P = \{a_n x^n + a_{n-1} x^{n-1} + \cdots + a_2 x^2 + a_1 x + a_0 \text{ | } a_i \in \mathbb{R}\}$$ với phép cộng và nhân đa thức tạo thành một vành. \pause
    
    Nếu giới hạn lại hệ số đa thức thành $a_i \in \mathbb{Q}$ hoặc $a_i \in \mathbb{Z}$ thì $P$ vẫn là vành
\end{frame}
\section{Trường (sơ lược)}
\begin{frame}{Nội dung}
    \tableofcontents[
        currentsection,
        % currentsubsection,
        % hideothersubsections,
        sectionstyle=show/shaded,
        subsectionstyle=show/show/hide
    ]
\end{frame}
\begin{frame}{Trường}
    \textbf{Định nghĩa.} \textit{Trường (Field)} $(F, +, \times)$ gồm tập hợp $F$ các phần tử và hai phép toán 2 ngôi là phép cộng ($+$) và phép nhân ($\times$) thoả mãn các tính chất sau:
    \begin{itemize}
        \item $(F, +, \times)$ là miền nguyên
        \item Tồn tại phần tử nghịch đảo của phép nhân: với mọi $a \in F$, $a \neq 0$, tồn tại $a^{-1} \in F$ sao cho $a \times a^{-1} = 1$
    \end{itemize}
\end{frame}

\begin{frame}{Trường. Ví dụ}
    Ta thường làm việc trên các trường hữu hạn như:
    
    \textbf{Ví dụ 1.} Trường hữu hạn $GF(p)$: Với $p$ là số nguyên tố, trường hữu hạn $GF(p)$ tính toán trên tập hợp các thặng dư modulo $p$ (tập hợp $\{0, 1, \cdots, p-1\}$)
    
    \textbf{Ví dụ 2.} Trường hữu hạn $GF(p^n)$: Với $p$ là số nguyên tố và $n$ là số nguyên dương, trường hữu hạn tính toán trên tập hợp $$ P = \{(a_1, a_2, \cdots, a_n) \text{ | } a_i \in GF(p)\}$$ cùng với các quy tắc nhất định khác.
\end{frame}

\section{Bài tập tổng kết}
\begin{frame}{Bài tập tổng kết}
    \textbf{Số nguyên Gauss (Gaussian integer)} là số phức có dạng $$z = a + bi$$ với $a, b \in \mathbb{Z}$, $i^2 = -1$. Phép cộng và phép nhân 2 số nguyên Gauss được thực hiện như số phức thông thường.
    
    Ví dụ với 2 số nguyên Gauss $z_1 = a_1 + b_1 i$ và $z_2 = a_2 + b_2 i$ thì
    \begin{align*}
        z_1 + z_2 & = (a_1 + b_1) + (a_2 + b_2) i \\
        z_1 z_2 & = (a_1 a_2 - b_1 b_2) + (a_1 b_2 + a_2 b_1)i
    \end{align*}
    Câu hỏi: Tập hợp các số nguyên Gauss với phép cộng và nhân như trên không phải là trường, vì sao? Để trở thành trường cần điều kiện gì?
\end{frame}

\begin{frame}
    \begin{center}
        \Huge
        Cám ơn mọi người đã lắng nghe. Chúc buổi tối vui vẻ.
    \end{center}
\end{frame}
\end{document}