\begin{frame}{Kiến thức nền tảng}
\begin{enumerate}
    \item Khoa học máy tính
    \begin{itemize}
        \item Lập trình (C/C++/Python)
        \item Cấu trúc dữ liệu và giải thuật (bruteforce)
        \item Mạng máy tính (giao thức bảo mật web, mạng)
    \end{itemize}
    \item Toán
    \begin{itemize}
        \item Đại số tuyến tính (Linear Algebra)
        \item Lý thuyết số (Number Theory)
        \item Toán trừu tượng (Abstract Algebra)
    \end{itemize}
\end{enumerate}
\end{frame}
\begin{frame}{Khoa học máy tính}
\begin{enumerate}
    \item Lập trình: khuyến khích sử dụng các ngôn ngữ scripting, code ngắn gọn, nhiều thư viện hỗ trợ, như Python, Ruby, Rust, ...
    \item Biểu diễn dữ liệu trên máy tính: hệ cơ số 10, 2, 8, 16
    \item Encode và Decode: base64, base32, ....
    \item Cấu trúc dữ liệu và giải thuật
    \begin{itemize}
        \item Bruteforce
        \item Chia nhị phân
        \item Danh sách, Cây nhị phân
    \end{itemize} 
    \item Mạng máy tính
    \begin{itemize}
        \item Lập trình socket (với Python)
        \item Giao thức HTTPS
    \end{itemize}
\end{enumerate}
\end{frame}
\begin{frame}{Toán I}
\begin{enumerate}
\item Đại số tuyến tính
\begin{itemize}
    \item Ma trận. Các phép toán trên ma trận. Định thức
    \item Không gian vector. Cơ sở. Chiều
    \item Ma trận trực giao. Ma trận trực chuẩn
    \item Chéo hóa ma trận
    \item ....
\end{itemize}
\item Lý thuyết số
\begin{itemize}
    \item Đồng dư thức modulo. Thuật toán tính $a^n \pmod p$ nhanh
    \item Ước chung. Bội chung. Thuật toán Euclid. Thuật toán Euclid mở rộng và nghịch đảo trong modulo $p$
    \item Phi hàm Euler. Định lý Euler. Định lý Fermat
    \item Định lý số dư Trung Hoa (Chinese Remainder Theorem)
    \item Số chính phương modulo $p$ nguyên tố (Quadratic Residue)
\end{itemize}
\end{enumerate}
\end{frame}
\begin{frame}{Toán II}
\begin{enumerate}
\item Toán trừu tượng
\begin{itemize}
    \item Lý thuyết nhóm. Định nghĩa. Tính chất của nhóm, vành, trường. Ví dụ và giải thích
\end{itemize}
\item Toán rời rạc
\begin{itemize}
    \item Đại số Boolean
\end{itemize}
\item Xác suất thống kê
\end{enumerate}
\end{frame}

\begin{frame}{Kỹ năng cần có}
    \begin{enumerate}
        \item Lập trình
        \item Đọc hiểu code
        \item Research về code, về toán, ....
    \end{enumerate}
\end{frame}