\begin{frame}
    \titlepage
\end{frame}

\begin{frame}{Nội dung}
\begin{enumerate}
    \item Giới thiệu mật mã học (Cryptography)
    \item Mật mã học có gì?
    \item Kiến thức nền tảng
    \begin{enumerate}
        \item Khoa học máy tính
        \item Toán
    \end{enumerate}
\item Luyện tập
    \end{enumerate}
\end{frame}
\begin{frame}{Giới thiệu}
    Mật mã học là lĩnh vực sử dụng các lý thuyết toán nhằm đảm bảo an toàn thông tin.
    
    Mật mã học được dùng để: \begin{enumerate}
        \item Khiến dữ liệu bình thường trở nên "không đọc được" và không bị kẻ tấn công, bên thứ ba, ... đọc trộm
        \item Tạo chữ ký điện tử để xác thực chủ sở hữu, thời hạn sử dụng, ..... cho các đối tượng điện tử (tài liệu, trang web, phần mềm, ....)
        \item ......
    \end{enumerate}
\end{frame}

\begin{frame}{Mật mã học có gì?}
\begin{enumerate}
    \item Mã hóa đối xứng cổ điển
    \item Mã hóa đối xứng hiện đại
    \item Mã hóa khóa công khai
    \item Mã chứng thực thông điệp. Hàm băm
    \item Chữ kí điện tử
    \item Giao thức
    \item Phương pháp trao đổi khóa
    \item \textbf{Các phương pháp phá mã}
\end{enumerate}
\end{frame}

\begin{frame}{Thành phần trong một hệ mật mã}
Hệ mật mã (cryptosystem) là 1 hệ thống gồm:

\begin{itemize}
    \item Bản rõ (Plaintext): dữ liệu bình thường, đọc được
    \item Bản mã (Ciphertext): dữ liệu không đọc được
    \item Mã hóa (Encrypt): 1 hàm $E(P, K_1)$ với $P$ là plaintext, $K_1$ là khóa (key) để chuyển dữ liệu đọc được thành không đọc được
    \item Giải mã (Decrypt): 1 hàm $D(C, K_2)$ với $C$ là ciphertext, $K_2$ là khóa (key) để chuyển dữ liệu đã bị mã hóa trở về dữ liệu đọc được ban đầu
    \item Key 1, Key 2: khóa, dùng để mã hóa và giải mã
\end{itemize}
    \begin{tikzpicture}[node distance=1.25cm]
    \node (pt1) [data] {Bản rõ (Plaintext)};
    \node (enc) [function, right of=pt1, xshift=1cm, text width=1cm] {Mã hóa};
    \node (ct) [data, right of=enc, xshift=1cm] {Bản mã (Ciphertext)};
    \node (dec) [function, right of=ct, xshift=1cm, text width=1cm] {Giải mã};
    \node (pt2) [data, right of=dec, xshift=1cm] {Bản rõ (Plaintext)};
    \node (key1) [data, below of=enc, yshift=-0.5cm] {Key 1};
    \node (key2) [data, below of=dec, yshift=-0.5cm] {Key 2};
    \draw [arrow] (pt1) -- (enc);
    \draw [arrow] (key1) -- (enc);
    \draw [arrow] (enc) -- (ct);
    \draw [arrow] (ct) -- (dec);
    \draw [arrow] (key2) -- (dec);
    \draw [arrow] (dec) -- (pt2);
    \end{tikzpicture}
\end{frame}

\begin{frame}{Phân loại hệ mật mã}
Theo sơ đồ trên:
\begin{enumerate}
    \item Nếu $K_1 = K_2$ gọi là mã đối xứng (Symmetric)
    \begin{itemize}
        \item Mã hóa cổ điển: Caesar, Vigenere, Hill, French Fence, ....
        \item Mã hóa hiện đại:
        \begin{itemize}
        \item Mã dòng: chia dữ liệu thành các đoạn có độ dài bằng nhau, xem các đoạn đó là 1 "dòng" và mã hóa dòng đó
        \item Mã khối: chia dữ liệu thành các khối có độ dài bằng nhau (khối cuối có thể ngắn hơn độ dài đó) và mã hóa theo các khối đó
    \end{itemize}
    \end{itemize}
    \item Nếu $K_1 \neq K_2$ gọi là mã bất đối xứng (Assymmetric) \\ Ví dụ: 
    \begin{itemize}
        \item RSA (hệ mã được sử dụng rộng rãi trong CTF lẫn thực tế)
        \item ElGamal
        \item ...
    \end{itemize}
\end{enumerate}
\end{frame}