\section{Введение}
Криптография появилась очень долго раньше, чтобы обслужить в войнах. Она использовала простую математику но очень эффективно для защиты информации. Когда компьютер и сети родились, криптография использовалась для защиты данные, но теперь сложные математические понятия используют, благодаря возможностей компьютеров.

\subsection{Что такое криптография и её назначения в области информационной безопасности}

	Что такое криптография? - это область науки, которая использует математику для защиты информация

Когда передавать данные на сетях, мы заботимся о

\begin{itemize}[label={--},noitemsep,nolistsep]
    \item безопасности: если А хочет передать информации в В, но не хочет другие знают, тогда А использует криптографию, чтобы трансфомировать обычные данные в данные, которые не могут читать
    \item авторизации: А хочет знает, послал ли В эти информации, или другой человек С послал но скажет себя В
\end{itemize}
\subsection{Основные понятия криптографических систем и требования к ним}
    
Основные понятия криптографического системы: \cite{lecturesOIB}

\begin{itemize}[noitemsep,nolistsep,label={--}]
	\item Шифрование - преобразовательный процесс: исходный текст, который носит также название открытого текста, заменяется шифрованным текстом. 
	\item Дешифрование - обратный шифрованию процесс. На основе ключа шифрованный текст преобразуется в исходный
	\item Ключ - информация, необходимая для беспрепятственного шифрования и дешифрования текстов
\end{itemize}

Требования к криптосисемам:

\begin{itemize}[noitemsep,nolistsep,label={--}]
	\item зашифрованное сообщение должно поддаваться чтению только при наличии ключа; 
	\item число операций, необходимых для определения использованного ключа шифрования по фрагменту шифрованного сообщения и соответствующего ему открытого текста, должно быть не меньше общего числа возможных ключей;
	\item число операций, необходимых для расшифровывания информации путем перебора всевозможных ключей, должно иметь строгую нижнюю оценку и выходить за пределы возможностей современных компьютеров (с учетом возможности использования сетевых вычис-лений);
	\item знание алгоритма шифрования не должно влиять на надежность защиты; 
	\item незначительное изменение ключа должно при водить к существенному изменению вида зашифрованного сообщения даже при использовании одного и того же ключа;
	\item структурные элементы алгоритма шифрования должны быть неизменными; 
	\item дополнительные биты, вводимые в сообщение в процесс е шифрования, должен быть полностью и надежно скрыты в шифрованном тексте;
	\item длина шифрованного текста должна быть равной длине исходного текста; 
	\item не должно быть простых и легко устанавливаемых зависимостей между ключами, последовательно используемыми в процессе шифрования;
	\item любой ключ из множества возможных должен обеспечивать надежную защиту информации; 
	\item алгоритм должен допускать как программную, так и аппаратную реализацию, при этом изменение длины ключа не должно вести к качественному ухудшению алгоритма шифрования
\end{itemize}
\subsection{Разделы криптографии}
\begin{itemize}[noitemsep,nolistsep,label={--}]
	\item \textbf{Симметричные криптосистемы} использует один и тот же ключ и для шифрования, и для расшифровывания. Алгоритм и ключ выбирается заранее и известен обеим сторонам. Сохранение ключа в секретности является важной задачей для установления и поддержки защищённого канала связи
	\item \textbf{Асимметричные криптосистемы} использует два разных ключа: один для шифрования (который также называется открытым), другой для расшифровывания (называется закрытым). Данные ключи связанны друг с другом определенным математическим образом
	% Открытый ключ передаётся по открытому (то есть незащищённому, доступному для наблюдения) каналу и используется для шифрования сообщения и для проверки электронной (цифровой) подписи. Для расшифровки сообщения и для генерации электронной (цифровой) подписи используется секретный ключ
	\item \textbf{Системы электронной подписи} - это присоединяемое к тексту его криптографическое преобразование, которое позволяет при получении текста другим пользователем проверить авторство и подлинность сообщения
	\item \textbf{Системы управления ключами} - это информационные системы, целью которых является составление и распределение ключей между пользователями информационной системы
	
\end{itemize}