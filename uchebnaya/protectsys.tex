\section{Криптографическая система защиты}
Алгоритмы с секретным ключом или с открытым ключом обладают свои достоинства и недостатки. Криптографические системы защита часто используют одновременно два типа системы. В этом участии мы рассмотрим криптографическую систему защиты и пример, который использует протокол HTTPS.

\subsection{Схема системы безопасности}

Пользователь 1 (Алис) и пользователь 2 (Боб) хотят передавать информацию между ними. Алис и Боб согласны один алгоритм с открытым ключом (функция шифрования $E_{pub}$ и её соответственная функция расшифрования $D_{pub}$) и один алгоритм с секретным ключом (функции шифрования и расшифрования соответственно $E_{priv}$ и $D_{priv}$). 

Алгоритм с открытым ключом используется, чтобы передать ключ для алгоритма с секретным ключом.

\begin{enumerate}[noitemsep,nolistsep]
	\item Алис создает секретный ключ $k_1$ и от этого секретного ключа, Алис вычисляет открытый ключ $k_2$.
	\item Алис посылает $k_2$ к Бобу.
	\item Боб создает ключ для алгоритм с секретным ключом $K$ и шифрует его с помощью ключа $k_2$ ($K_e=E_{pub}(k_2, K)$).
	\item Боб посылает $K_e$ к Алису.
	\item Алис вычисляет $K$ по формуле $K = D_{pub}(k_1, K_e)$.
	\item Теперь Алис и Боб обладают ключом $K$, они могут передавать любое сообщение $m$ по формулам $c=E_{priv}(K, m)$ и $m=D_{priv}(K, c)$. 
\end{enumerate}
	
\begin{table}[ht]
    \caption{Схема системы безопасности}
	\centering
	\begin{tabularx}{\textwidth}{
			| >{\centering\arraybackslash}X |
			| >{\centering\arraybackslash}X |
		}
		\hline
		\textbf{Алис} & \textbf{Боб} \\
		\hline\hline
		\multicolumn{2}{|c|}{\textit{Создание ключом для алгоритма с отрытым ключом}} \\
		\hline
		Создание секретный ключ $k_1$ и вычисляет открытый ключ $k_2$ & \\
		\hline
		\multicolumn{2}{|c|}{Передача $k_2$ к Бобу} \\
		\hline\hline
		\multicolumn{2}{|c|}{\textit{Передача ключ для алгорима с секретным ключом}} \\
		\hline
		& Создание ключ $K$ и шифрование его $K_e = E_{pub}(k_2, K)$ \\
		\hline
		\multicolumn{2}{|c|}{Передача $K_e$ к Алису} \\
		\hline
		Расшифрование ключ по формуле $K=D_{pub}(k_1, K_e)$ &  \\
		\hline\hline
		\multicolumn{2}{|c|}{\textit{Передачи сообщений}} \\
		\hline
		Сообщение $m$ и шифрование $c=E_{priv}(K, m)$ & \\
		\hline
		& Расшифрование $m=D_{priv}(K, c)$ \\
		\hline
	\end{tabularx}
\end{table}

\subsection{Схема системы авторизации}

Клиент Алис хочет передавать сообщении с сервером Боб, но не уверен, подделывает ли кто-то сервера. Этот случай называется фишинговой атакой. Для защита от фишинг-атаки, используют эту схему

\begin{enumerate}[noitemsep,nolistsep]
	\item Одна доверенная сторона, которая называется центром сертификации (англ - certificate authority, СА) создает цифровую сертификацию для сервера.
	\item В сертификации есть открытый ключ, информация о владельце, и цифровая подпись.
	\item Клиент хочет обменять информации с сервером. Начало, он спросит сервера о сертификации и проверит цифровую подпись, если действительный, клиент начинает обменяет данные с сервером. Если недействительный, клиент кончится.
\end{enumerate}

\begin{table}[ht]
    \caption{Схема системы авторизации}
	\centering
	\begin{tabularx}{\textwidth}{
			| >{\centering\arraybackslash}X |
			| >{\centering\arraybackslash}X |
		}
		\hline
		\textbf{Сервер} & \textbf{Клиент} \\
		\hline
		Получить цифровую сертификацию из доверенной стороны & \\
		\hline
		& Получить сертификацию из сервера \\
		\hline
		& Проверить сертификацию \\
		\hline
	\end{tabularx}
	
\end{table}

\subsection{Криптографическая система защиты по протоколу HTTPS}

HTTP - это протокол для передачи файлы (как html, рисунки, ...) между компьютерами на сети. Но все данные представляют себя в виде \textit{незашифрованный текст}, поэтому рискованно. С помощью протокол SSL/TLS, файлы зашифруются для безопасности и авторизации. Тогда HTTPS = HTTP + SSL/TLS

Теперь рассмотрим основные этапы протокола HTTPS между сервером и клиентом.

Этап 1: проверка сервера и обмен ключ для шифрования сообщений между сервером и клиентом.
	\begin{itemize}[label={--},noitemsep,nolistsep]
		\item Сервер зарегистрирует цифровую сертификацию из доверенной стороны, например, Cloudflare.
		\item В сертификации есть цифровая попись, алгоритм для проверки цифровой подписи (например, ECDSA), алгоритм для шифрования (например, AES 256 бит), алгоритм для обмена ключа (например, RSA).
		\item Клиент посылает сообщение в сервер для начала обмена информации, и ждёт ответ от сервера.
		\item Сервер тогда посылает сертификацию и открытый ключ (алгоритм обмена ключа) в клиент
		\item Клиент использует алгоритм для проверки подписи (ECDSA), чтобы проверить сертификацию. Если сертификация не действительная или истекшая, клиент закончит передавать информации.
		\item Потом, клиент генерирует ключ для шифрования (AES 256 бит) и зашифрует его с помощью открытого ключа алгоритма обмена ключа (RSA).
		\item Клиент посылает шифрованный ключ в сервер.
	\end{itemize}
Этап 2: передача сообщений HTTP c помощью криптографии.
	\begin{itemize}[label={--},noitemsep,nolistsep]
		\item Сервер расшифрует шифрованный ключ, который получит от клиента.
		\item Сервер зашифрует каждое сообщение HTTP, которое клиент попросит, по этому ключу и посылает шифрованное сообщение.
		\item Клиент использует этот ключ, чтобы расшифровать это шифрованное сообщение и покажет на экран.
	\end{itemize}
Простой пример HTTPS, использующий ECDSA, AES 256 бит и RSA, показан в рисунке 4. 
\begin{figure}[!h]
	\centering
	\begin{tikzpicture}[node distance=2cm]
		\tikzstyle{process} = [rectangle, minimum width=3cm, minimum height=1cm, text centered, draw=black, fill=white!30]
		\tikzstyle{decision} = [diamond, minimum width=3cm, minimum height=1cm, text centered, text width=3cm, draw=black, fill=white!30]
		\tikzstyle{blank} = [rectangle, minimum width=3cm, minimum height=1cm, draw=white]
		\tikzstyle{arrow} = [thick,->,>=stealth]
		
		\node (server) [process] {Сервер};
		\node (internet) [blank, right of=server, xshift=3cm] {Интернет};
		\node (client) [process, right of=internet, xshift=3cm] {Клиент};
		\node (bl1) [blank, below of=server, yshift=-1cm] {};
		\node (trust) [process, below of=client, yshift=-1cm] {Дорверить Cloudflare};
		\node (verify) [process, below of=trust] {Проверить сертификацию};
		\node (valid) [process, below of=verify, yshift=-0.5cm] {\begin{tabular}{c}Поключение \\ ключа AES и \\ шифрование его\end{tabular}};
		\node (deckey) [process, below of=server, yshift=-5.5cm] {\begin{tabular}{c}Расшифрования \\ шифрованного ключа\end{tabular}};
		\node (recover) [process, below of=deckey] {Ключ AES};
		\node (bl3) [blank, below of=valid, yshift=-2.5cm] {};
		\node (http) [process, below of=recover, yshift=-0.5cm] {\begin{tabular}{c}нормальное HTTP \\ сообщение для \\ передечи index.html\end{tabular}};
		\node (https) [process, below of=http, yshift=-1cm] {\begin{tabular}{c}
				зашифрование этого \\ сообщеня HTTP
		\end{tabular}};
		\node (plain) [process, below of=bl3, yshift=-1cm] {\begin{tabular}{c}расшифрование \\ шифрованного \\ сообщения\end{tabular}};
		\node (start) [process, below of=plain, yshift=-1cm] {\begin{tabular}{c}файл index.htm \\ показан на экране\end{tabular}};
		
		\draw [arrow] (bl1) -- node [above] {сертификация} node [below] {открытый ключ RSA} (trust);
		\draw [arrow] (valid) -- node [above] {посылка} node[below]  {шифрованный ключ} (deckey);
		\draw [arrow] (bl3) -- node [above] {нужен файл inhdex.html} (http);
		\draw [arrow] (https) -- node [above] {интернет} (plain);
		\draw [arrow] (trust) -- (verify);
		\draw [arrow] (verify) -- (valid);
		\draw [arrow] (deckey) -- (recover);
		\draw [arrow] (recover) -- (http);
		\draw [arrow] (http) -- (https);
		\draw [arrow] (https) -- (plain);
		\draw [arrow] (plain) -- (start);
	\end{tikzpicture}
	\caption{Простая блок-схема протокола HTTPS}
\end{figure}
