\section{Криптографическая система (криптосистема)}

Что такое криптографическая система защита? Криптосистема представляет собой неинтерактивный протокол конфиденциальной передачи сообщений. Имеются два основных типа криптосистем — с секретным и с открытым ключом. (cryptography.ru)

Пусть $K$ - пространство ключом. $k_1$, $k_2$ - ключи шифрования и расшифрования соответственно. $E$ - функция шифрования для произвольного ключа $k_1 \in K$, что $E(k_1, m) = c$. Здесь $c \in C$, где $C$ - пространство шифротекстов, а $m \in M$ - пространство сообщений

$D$ - функция расшифрования, с помощью которой можно найти исходное сообщение $m$, зная шифротекст $c$: $D(k_2, c) = m$.
 
\begin{figure}[ht]
	\centering
	\begin{tikzpicture}[node distance=2cm]
		\node (user1) [block] {Пользователь 1}; 
		\node (pt1) [block, below of=user1] {Сообщение $m$};
		\node (enc) [block, below of=pt1, yshift=-1.5cm] {\begin{tabular}{c}Шифрование \\ $c=E(k_1, m)$\end{tabular}};
		\node (key1) [block, below of=enc, yshift=-1.5cm] {Ключ $k_1$};
		\node (user2) [block, right of=user1,xshift=8cm] {Пользователь 2};
		\node (pt2) [block, below of=user2] {Начальное сообщение $m$};
		\node (dec) [block, below of=pt2, yshift=-1.5cm] {\begin{tabular}{c}Расширование \\ $m=D(k_2, c)$\end{tabular}};
		\node (key2) [block, below of=dec, yshift=-1.5cm] {Ключ $k_2$};
		
		\node (aa) [dottedblock, fit = (pt1) (enc) (key1)] {};
		\node (bb) [dottedblock, fit = (pt2) (dec) (key2)] {};
		\draw [arrow] (pt1) -- (enc);
		\draw [arrow] (key1) -- (enc);
		\draw [arrow] (enc) -- node[above] {Открытный} node [below] {каналь} (dec);
		\draw [arrow] (key2) -- (dec);
		\draw [arrow] (dec) -- (pt2);
	\end{tikzpicture}
	\caption{Блок-схема криптографической системы}
\end{figure}
\subsection{Криптосистема с секретным ключом}

Если $k_1 = k_2$, эта криптосистема называется симметричной (криптосистема с секретным ключом)

\textbf{Достоинства симметричной криптосистемы}
\begin{itemize}[label={--},noitemsep,nolistsep]
	\item Быстро: симметричная криптосистема часто быстро, потому что ключи не длинные, но достаточно, чтобы  обеспечить безопасность
	\item Сохранение ключи не требует много пространств
\end{itemize}

\textbf{Недостатки симметричной криптосистемы}
\begin{itemize}[label={--},noitemsep,nolistsep]
	\item Симметричная криптосистема требует способы, чтобы передавать ключи между пользователями. В этом случае используют способы обмена ключами или асимметричную криптосистему.
	\item Получатель не знает, подделывает ли кто-нибудь отправителя, или обратно, отправитель не знает, подделывает ли кто-нибудь получателя. Этот случай называется Man-In-The-Middle-Attack (MITM Attack).
\end{itemize}

\begin{figure}[ht]
	\centering
	\subfigure[нормальный режим симметричной криптосистемы]
	{
		\begin{tikzpicture}[node distance=2cm,->,every edge/.style={font=\footnotesize, draw}]
			\node (sender) [block] at (0, 0) {Отправитель}; 
			\node (receiver) [block, right of=sender, xshift=4cm] {Получатель}; 
			
			\draw[implies-implies, double equal sign distance] (sender) -- node [above] {Ключ $K$} (receiver);
		\end{tikzpicture}
	}
	
	\subfigure[Man-In-The-Middle Attack]
	{
		\begin{tikzpicture}[node distance=2cm,->,every edge/.style={font=\footnotesize, draw}]
			\node (sender) [block] at (0, 0) {Отправитель}; 
			\node (hacker) [block, right of=sender, xshift=4cm] {Хакер};
			\node (receiver) [block, right of=hacker, xshift=4cm] {Получатель}; 
			
			\draw[implies-implies, double equal sign distance] (sender) -- node [above] {Ключ $K$} (hacker);
			\draw[implies-implies, double equal sign distance] (hacker) -- node [above] {Ключ $K'$} (receiver);
		\end{tikzpicture}
	}
	
	\caption{Симметричная криптографическая система}

\end{figure}

\subsection{Алгоритмы используемых в криптографических системах с секретным ключом}

\begin{itemize}[label={--},noitemsep,nolistsep]
	\item DES (\textit{Data Encryption Standard}) - симметричный алгоритм шифрования, в котором один ключ используется как для шифрования, так и для расшифрования данных. DES разработан фирмой IBM и утвержден правительством США в 1977 году как официальный стандарт (FIPS 46-3). DES имеет блоки по 64 бит и 16 цикловую структуру сети Фейстеля, для шифрования использует ключ с длиной 56 бит (в преобразовании учавствуют раундовые ключи по 48 бит). Алгоритм использует комбинацию нелинейных (S-блоки) и линейных (перестановки E, IP, IP-1) преобразований.
	\item AES (\textit{Advanced Encryption Standard}) - симметричный алгоритм блочного шифрования (размер блока 128 бит, ключ 128/192/256 бит), принятый в качестве стандарта шифрования правительством США по результатам конкурса AES. Американский стандарт, опубликованный в 2001 году. В современный криптографических подуктах, наверно, не найдется таких, которые бы не использовали AES. Используется в WI-FI, WinRAR, PGP. DES- предшественник AES. \cite{bmstucrypto}
\end{itemize}

Информации об алгоритме AES
\begin{itemize}[label={--},noitemsep,nolistsep]
	\item Алгоритм победил: Rijndael

	\item Авторы: Vincent Rijmen, Joan Daemon 

	\item Характеристики AES: блочный шифр; размер блока 128 бит; размер ключа 128 бит (затем 192/256); эффективная программная и аппаратная реализация; возможность эффективной реализации на 32-ъх разрядных CPU; простота архитектура; отсутствие ограничений на использование
\end{itemize}

\subsection{Криптосистема с открытым ключом}
Когда $k_1 \neq k_2$, эта криптосистема называется асимметричной (криптосистема с открытым ключом)

$k_1$ - открытый ключ, который используется для шифровария и все пользователи знают.

$k_2$ - секретный ключ, который используется для расшифрования и только получатель знает

Ключи $k_1$ и $k_2$ часто имеют математическое отношение. Из секретного ключа $k_2$ мы просто вычисляем открытый ключ $k_1$, но из открытого ключа очень трудно (или невозможно) вычисляем секретный ключ.

\textbf{Достоинства асимметричной криптосистемы}
\begin{itemize}[label={--},noitemsep,nolistsep]
	\item Безопасность: только пользователь, который имеет секретный ключ, может расшифровать сообщения. Поэтому если хакер (или третья партия) как-нибудь возьмет сообщению, он (хакер) не может расшифровать и читает её (не может использовать открытый ключ для расшифрования).
	\item Авторизация: если мы изменяем роли двух ключа, значит отправитель обладает секретным ключом и получатель обладает открытым ключом, теперь нет безопасности, потому что все пользователи знают открытый ключ и могут расшифровать. Но мы получаем важную характеристику - мы знаем точно, кто посылает эту сообщению и он не может отрицать, потому что никто кроме его обладает этим ключом. Эта характеристика используется для создания цифровой подписи (Блок-схема цифровой подписи покажется на рисунке 3).
\end{itemize}

\textbf{Недостатки}
\begin{itemize}[label={--},noitemsep,nolistsep]
	\item Медленно: асимметричная криптосистема часто работает с огромными числами для целя безопасности.
\end{itemize}

\begin{figure}[!htb]
	\centering
	\begin{tikzpicture}[node distance=2cm]
		\tikzstyle{startstop} = [rectangle, rounded corners, minimum width=3cm, minimum height=1cm,text centered, draw=black, fill=red!30]
		\tikzstyle{io} = [trapezium, trapezium left angle=70, trapezium right angle=110, minimum width=3cm, minimum height=1cm, text centered, draw=black, fill=blue!30]
		\tikzstyle{process} = [rectangle, minimum width=3cm, minimum height=1cm, text centered, draw=black, fill=orange!30]
		\tikzstyle{decision} = [diamond, minimum width=3cm, minimum height=1cm, text centered, draw=black, fill=white!30]
		\tikzstyle{arrow} = [thick,->,>=stealth]
		
		\node (owner) [block] {Владенец};
		\node (key) [block, below of=owner, yshift=-2cm] {\begin{tabular}{c}Секретный ключ $k_2$ \\ $+$ \\ Документ $m$\end{tabular}};
		\node (sign) [block, below of=key, yshift=-2cm] {\begin{tabular}{c}подпись \\ $s = Sign(m, k_2)$\end{tabular}};
		
		
		\node (user1) [block, right of=owner, xshift=5cm] {Пользователь};
		\node (key1) [block, below of=user1, yshift=-2cm] {\begin{tabular}{c}
				Открытый ключ $k_1$ \\ $+$ \\ Документ $m$
		\end{tabular}};
		\node (ver1) [decision, below of=key1, yshift=-2cm] {Проверить $s$};
		\node (true1) [block, right of=ver1, xshift=2cm] {Правильно};
		\node (false1) [block, below of=ver1, yshift=-2cm] {Неверно};
		
		\draw [arrow, line width=1mm] (key) -- (sign);
		\draw [arrow, line width=1mm] (key1) -- (ver1);
		\draw [arrow, line width=1mm] (ver1) -- (true1);
		\draw [arrow, line width=1mm] (ver1) -- (false1);
		
		\node (own) [dottedblock, fit = (key) (sign)] {};
		\node (use) [dottedblock, fit = (key1) (ver1) (true1) (false1)] {};
	\end{tikzpicture}
	\caption{Процессы создания и проверка цифровой подписи}
\end{figure}
\subsection{Алгоритмы используемых в криптографических системах с открытым ключом}
\begin{itemize}[label={--},noitemsep,nolistsep]
	\item RSA (\textit{Rivest-Shamir-Adleman}) - криптографический алгоритм с открытым ключом, основывающийся на вычислительной сложности задачи факторизации больших целых чисел. Криптосистема RSA стала первой системой, пригодной и для шифрования, и для цифровой подписи. Алгоритм используется в большом числе криптографических приложений, включая PGP, S/MIME, TLS/SSL, IPSEC/IKE и других.
	\item Схема Эль-Гамаля (Elgamal) - криптосистема с открытым ключом, основанная на трудности вычисления дискретных логарифмов в конечном поле. Криптосистема включает в себя алгоритм шифрования и алгоритм цифровой подписи. Схема была предложена Тахером Эль-Гамалем в 1985 году. Схема Эль-Гамаля лежит в основе бывших стандартов электронной цифровой подписи в США (DSA) и России (ГОСТ Р 34.10-94)
\end{itemize}

\textbf{RSA: алгоритм создания открытого и секретного ключей, алгоритмы шифрования и расширования} \cite{introcrypto}
\begin{itemize}[label={--},noitemsep,nolistsep]
	\item Выбираются два различных случайных простых числа $p$ и $q$ заданного размера (например, 1024 бита каждое);
	\item Вычисляется их произведение $n=p \cdot q$. Оно называется \textit{модулем};
	\item Вычисляется значения функции Эйлера от числа $n$: $\phi(n) = (p-1)\cdot(q-1)$;
	\item Выбирается целое число $e$ ($1 < e < \phi(n))$, взаимно простое со значением функции $\phi(n)$; \\ число $e$ называется \textit{открытой экпонентой};
	\item Вычисляется число $d$, мультипликативно обратное к числу $e$ по модулю $\phi(n)$, значит $d \cdot e = 1 \pmod{\phi(n)}$ \\ число $d$ называется \textit{секретной экспонентой};
	\item Пара $(e, n)$ публикуется в качестве \textit{открытого ключа RSA}
	\item Пара $(d, n)$ играет роль \textit{закрытого ключа RSA};
	\item Шифрование: у нас есть сообщение $m$ ($1 < m < n$), тогда его соответственный шифротекст вычисляется по формуле \\ $c = m^e \pmod n$
	\item Расшифрование: с шифротекстом $c$, мы возвратим начальное сообщение $m = c^d \pmod n$
\end{itemize}

\textbf{Elgamal: алгоритм создания открытого и секретного ключей, алгоритмы шифрования и расшифрования} \cite{introcrypto}
\begin{itemize}[label={--},noitemsep,nolistsep]
	\item Одна доверенная сторона выбирает и публикует огромное простое число $p$ и элемент $g \bmod p$ с большим (простым) ордером.
	\item Алис выбирает секретный ключ $a$ ($1 \leq a \leq p-1$) и вычисляет $A = g^a \pmod p$ \\ Число $A$ - отрытый ключ
	\item Шифрование: Боб хочет шифровать сообщение $m$ ($1 \leq m \leq p-1$). Он выбирает случайное число $k$ и использует открытый ключ Алиса $A$, чтобы вычислять \\ $c_1 = g^k \pmod p$ и $c_2 = mA^k \pmod p$ \\ Шифротекст будет парой $(c_1, c_2)$
	\item Расшифрование: Алис получит начальное сообщение $m = (c_1^2)^{-1} \cdot c_2 \pmod p$
\end{itemize}

